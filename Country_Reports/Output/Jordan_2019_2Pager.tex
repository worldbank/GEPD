\PassOptionsToPackage{unicode=true}{hyperref} % options for packages loaded elsewhere
\PassOptionsToPackage{hyphens}{url}
%
\documentclass[twocolumn]{article}
\usepackage{lmodern}
\usepackage{setspace}
\setstretch{1.0}
\usepackage{amssymb,amsmath}
\usepackage{ifxetex,ifluatex}
\usepackage{fixltx2e} % provides \textsubscript
\ifnum 0\ifxetex 1\fi\ifluatex 1\fi=0 % if pdftex
  \usepackage[T1]{fontenc}
  \usepackage[utf8]{inputenc}
  \usepackage{textcomp} % provides euro and other symbols
\else % if luatex or xelatex
  \usepackage{unicode-math}
  \defaultfontfeatures{Ligatures=TeX,Scale=MatchLowercase}
\fi
% use upquote if available, for straight quotes in verbatim environments
\IfFileExists{upquote.sty}{\usepackage{upquote}}{}
% use microtype if available
\IfFileExists{microtype.sty}{%
\usepackage[]{microtype}
\UseMicrotypeSet[protrusion]{basicmath} % disable protrusion for tt fonts
}{}
\IfFileExists{parskip.sty}{%
\usepackage{parskip}
}{% else
\setlength{\parindent}{0pt}
\setlength{\parskip}{6pt plus 2pt minus 1pt}
}
\usepackage{hyperref}
\hypersetup{
            pdfborder={0 0 0},
            breaklinks=true}
\urlstyle{same}  % don't use monospace font for urls
\usepackage[left=0.5cm,right=0.5cm,top=1.5cm,bottom=1.2cm]{geometry}
\usepackage{graphicx,grffile}
\makeatletter
\def\maxwidth{\ifdim\Gin@nat@width>\linewidth\linewidth\else\Gin@nat@width\fi}
\def\maxheight{\ifdim\Gin@nat@height>\textheight\textheight\else\Gin@nat@height\fi}
\makeatother
% Scale images if necessary, so that they will not overflow the page
% margins by default, and it is still possible to overwrite the defaults
% using explicit options in \includegraphics[width, height, ...]{}
\setkeys{Gin}{width=\maxwidth,height=\maxheight,keepaspectratio}
\setlength{\emergencystretch}{3em}  % prevent overfull lines
\providecommand{\tightlist}{%
  \setlength{\itemsep}{0pt}\setlength{\parskip}{0pt}}
\setcounter{secnumdepth}{0}
% Redefines (sub)paragraphs to behave more like sections
\ifx\paragraph\undefined\else
\let\oldparagraph\paragraph
\renewcommand{\paragraph}[1]{\oldparagraph{#1}\mbox{}}
\fi
\ifx\subparagraph\undefined\else
\let\oldsubparagraph\subparagraph
\renewcommand{\subparagraph}[1]{\oldsubparagraph{#1}\mbox{}}
\fi

% set default figure placement to htbp
\makeatletter
\def\fps@figure{htbp}
\makeatother

\usepackage{fancyhdr} \usepackage{booktabs,xcolor} \pagestyle{fancy} \renewcommand{\headrulewidth}{0pt} \rhead{2019} \fancypagestyle{plain}{\pagestyle{fancy}} \setlength{\headheight}{77.3pt} \setlength{\footskip}{35.1pt} \setlength{\textheight}{0.90\textheight} \pagenumbering{gobble} \usepackage[fontsize=9pt]{scrextend} \usepackage{float} \restylefloat{table} \usepackage{xcolor} \usepackage{multicol} \usepackage{array} \usepackage{colortbl} \usepackage{multirow} \usepackage{collcell} \usepackage{setspace} \usepackage{arydshln} \usepackage{caption} \captionsetup{skip=0pt} \setlength{\columnsep}{1.5cm}

\author{}
\date{\vspace{-2.5em}}

\begin{document}

\newcommand{\greynote}[1]{
    {\scriptsize
    \textcolor{darkgray}{\textit{Source:} #1}
  }
}

\newcommand{\greysource}[1]{
    {\scriptsize
    \textcolor{darkgray}{\textit{Source:} #1}
  }
}

\newcommand{\greydisclaimer}[1]{
    {\scriptsize
    \textcolor{darkgray}{\textit{Disclaimer:} #1}
  }
}

\newcommand*{\tabindent}{\hspace{1mm}}

\hypertarget{introduction}{%
\subsubsection{\texorpdfstring{\textbf{INTRODUCTION}}{INTRODUCTION}}\label{introduction}}

\textbf{The Global Education Policy Dashboard (GEPD): An innovative tool
to measure drivers of learning outcomes in basic education}\\
GEPD uses 3 data collection instruments to report on nearly 40
indicators that operationalize the World Development Report 2018
framework to track 3 areas for progress in education- Practices,
Policies, and Politics. Using these indicators, the dashboard highlights
areas where countries need to act to improve learning outcomes and
allows a way for governments to track progress as they act to close gaps
in these areas. For more information on GEPD, please visit
\textbf{\href{https://www.worldbank.org/en/topic/education/brief/global-education-policy-dashboard}{www.worldbank.org/global-education-policy-dashboard}}

\hypertarget{figure-1.-gepd-framework-practices-policies-and-politics-expanding-on-wdr-2018-framework}{%
\paragraph{Figure 1. GEPD framework (practices, policies and politics),
expanding on WDR 2018
framework}\label{figure-1.-gepd-framework-practices-policies-and-politics-expanding-on-wdr-2018-framework}}

\begin{center}\includegraphics[width=0.8\linewidth]{C:/Users/wb469649/OneDrive - WBG/Documents/Github/GEPD/Country_Reports/Data/full_circle} \end{center}

\hypertarget{instruments-of-data}{%
\subsubsection{\texorpdfstring{\textbf{INSTRUMENTS OF
DATA}}{INSTRUMENTS OF DATA}}\label{instruments-of-data}}

The \textbf{School Survey} consists of 8 modules to collect data across
200-300 schools on practices (the quality of service delivery in
schools) and de facto policy indicators. It consists of streamlined
versions of existing instruments together with new questions to fill
gaps in those instruments.\\
The \textbf{Policy Survey} collects information via interviews with
\textasciitilde{}200 officials per country at federal and regional level
to feed into the policy de jure indicators and identify key elements of
the policy framework.\\
The \textbf{Survey of public officials} collects information about the
capacity and orientation of the bureaucracy and political factors
affecting education outcomes. This survey is an education-focused
version of the civil-servant surveys from the Bureaucracy Lab, WBG.

\hypertarget{key-takeaways-jordan-2019}{%
\subsubsection{\texorpdfstring{\textbf{KEY TAKEAWAYS, JORDAN,
2019}}{KEY TAKEAWAYS, JORDAN, 2019}}\label{key-takeaways-jordan-2019}}

\begin{itemize}
\tightlist
\item
  52\% learning poverty is observed at grade 4 level
\item
  Overall Grade 4 student proficiency (\textgreater{}80\% student
  knowledge) is only 4\%, attributed to low numeracy proficiency 2\%).
\item
  Low teacher content knowledge attributed to poor teacher support and
  lack of strong instructional leadership (especially principal feedback
  on classroom observation)
\item
  Low capacity for learning in Grade 1 (particularly socio-emotional
  knowledge) is attributed to low enrolment in early childhood programs
  (13\%)
\item
  A \textasciitilde{}0.8 point difference exists in the policy de-jure
  and de-facto indicators on a 5 point scale for teaching and school
  management. Major gaps are seen in teacher support, teacher motivation
  and evaluation of school management
\item
  Bureaucracy scores high on mandates and accountability but low on
  impartial decision making
\end{itemize}

\setlength\dashlinedash{0.2pt}
\setlength\dashlinegap{1.5pt}
\setlength\arrayrulewidth{0.3pt}

\hypertarget{table-1.-key-gepd-outcome-and-practice-indicators} \\\cdashline{1-2}
Proficiency by End of Primary & {\cellcolor{red!15}52\%} \\\cdashline{1-2}
Proficiency on GEPD Assessment & {\cellcolor{red!15}4.4\%} \\\cdashline{1-2}
\hspace{1mm}\emph{Literacy proficiency} & {\cellcolor{red!15}22\%} \\\cdashline{1-2}
\hspace{1mm}\emph{Numeracy proficiency} & {\cellcolor{red!15}2\%} \\\cdashline{1-2}
Proficiency by Grade 2/3 & {\cellcolor{red!15}-999\%} \\\cdashline{1-2}
Net Adjusted Enrollment Rate & {\cellcolor{red!15}-999\%} \\\hline
\end{tabular}}
\end{table}
\raggedbottom

\begin{table}[H]
\resizebox{\columnwidth}{!}{\begin{tabular}{p{0.3\textwidth}cc}
\textbf{Indicator} & \textbf{Male} & \textbf{Female} \\ \hline
Grade 1 knowledge & {\cellcolor{yellow!15}74\%} & {\cellcolor{yellow!15}75\%}\\\cdashline{1-3}
Teacher Content Proficiency  & {\cellcolor{red!15}32\%} & {\cellcolor{red!15}35\%}\\\cdashline{1-3}
Student attendance & {\cellcolor{yellow!15}78\%} & {\cellcolor{green!15}82\%}\\\cdashline{1-3}
Teacher attendance & {\cellcolor{green!15}92\%} & {\cellcolor{green!15}93\%}\\\hline
\end{tabular}}
\\
\setstretch{0.8}\color{darkgray}\scriptsize{\textit{Source:} UIS, GLAD, GEPD, World Bank, Jordan, 2019. For information on indicators, please consult the World Bank \href{https://github.com/worldbank/GEPD}{\underline{GEPD}}, \href{https://github.com/worldbank/GLAD}{\underline{GLAD}} and \href{https://github.com/worldbank/LearningPoverty}{\underline{Learning Poverty}} repositories.}\\
\setstretch{0.8}\color{darkgray}\scriptsize{\textit{Notes:} All indicators are on a scale of 0-5 unless measured in \%. Green indicates indicator 'on-target', yellow indicates 'requires caution', red indicates 'needs improvement'}
\end{table}
\raggedbottom

\hypertarget{overall-learning-outcomes-52-learning-poverty-in-grade-4}{%
\subsubsection{\texorpdfstring{\textbf{OVERALL LEARNING OUTCOMES: 52\%
LEARNING POVERTY IN GRADE
4}}{OVERALL LEARNING OUTCOMES: 52\% LEARNING POVERTY IN GRADE 4}}\label{overall-learning-outcomes-52-learning-poverty-in-grade-4}}

Learning poverty is 11\% points better than the average for Middle East
\& North Africa (excluding high income) region and 23\% points worse
than the average for Upper middle income countries.

\hypertarget{figure-2.-leaning-poverty-jordan}{%
\paragraph{Figure 2. Leaning poverty,
Jordan}\label{figure-2.-leaning-poverty-jordan}}

\includegraphics{C:/Users/wb469649/ONEDRI~1/DOCUME~1/Github/GEPD/COUNTR~1/Output/JORDAN~1/figure-latex/lp_fig-1.pdf}

{\scriptsize
    \textcolor{darkgray}{\textit{Source:} Grey circles represent all countries. Jordan is marked as JOR. 'LAC' and 'UMC' indicate the averages of Jordan's region and income group}
  }

Average proficiency in Grade 4 is 20 points higher in literacy compared
to numeracy,2.1 points lower for boys compared to girls, and 2 points
lower in urban areas compared to rural areas.\\
\vfill\null

\hypertarget{figure-3.-grade-4-proficiency-jordan}{%
\paragraph{Figure 3. Grade 4 proficiency,
Jordan}\label{figure-3.-grade-4-proficiency-jordan}}

\includegraphics{C:/Users/wb469649/ONEDRI~1/DOCUME~1/Github/GEPD/COUNTR~1/Output/JORDAN~1/figure-latex/learning_figure-1.pdf}

\hypertarget{comparsion-between-de_facto-practice-indicatrs-and-policy-lever-indicators-that-aid-practices-for-teaching-and-school-management}{%
\subsubsection{\texorpdfstring{\textbf{COMPARSION BETWEEN DE\_FACTO
PRACTICE INDICATRS AND POLICY LEVER INDICATORS THAT AID PRACTICES FOR
TEACHING AND SCHOOL
MANAGEMENT}}{COMPARSION BETWEEN DE\_FACTO PRACTICE INDICATRS AND POLICY LEVER INDICATORS THAT AID PRACTICES FOR TEACHING AND SCHOOL MANAGEMENT}}\label{comparsion-between-de_facto-practice-indicatrs-and-policy-lever-indicators-that-aid-practices-for-teaching-and-school-management}}

\hypertarget{table-3.-teacher-effectiveness-jordan-2019}   & Attraction                                & \cellcolor{yellow!15}3.5 \\\cdashline{1-4} 
\hspace{1mm}\emph{Maths proficiency}               & {\cellcolor{red!15}58\%} &                                           & \cellcolor{yellow!15}\\\cdashline{1-2}   
\hspace{1mm}\emph{Language proficiency}            & {\cellcolor{red!15}25\%} & \multirow{-2}{*}{Selection \& deployment} & \multirow{-2}{*}{\cellcolor{yellow!15}3.5} \\\cdashline{1-4}        
Pedagogical skills                               & {\cellcolor{yellow!15}68\%}   &                                           & \cellcolor{red!15}\\\cdashline{1-2}
\hspace{1mm}\emph{\% Classroom culture}            & {\cellcolor{yellow!15}72\%} & \multirow{-2}{*}{Support}                 & \multirow{-2}{*}{\cellcolor{red!15}2.6} \\\cdashline{1-4}
\hspace{1mm}\emph{\% Instruction practices}        & {\cellcolor{yellow!15}74\%} &                                           & \cellcolor{yellow!15}\\\cdashline{1-2}
\hspace{1mm}\emph{\% Socio-emotional skills}       & {\cellcolor{yellow!15}68\%} & \multirow{-2}{*}{Evaluation}              & \multirow{-2}{*}{\cellcolor{yellow!15}3.6} \\\cdashline{1-4}
& \cellcolor{green!15} & Monitoring \& Accountability  & \cellcolor{yellow!15}3.1 \\\cdashline{3-4}
\multirow{-2}{*}{Teacher Attendance}             & \multirow{-2}{*}{\cellcolor{green!15}93\%} & Intrinsic motivation    & \cellcolor{yellow!15}3.4 \\\hline
\end{tabular}}
\\
\setstretch{0.8}\color{darkgray}\scriptsize{\textit{Notes:} Capacity for learning indicates \% students with knowledge\textgreater{80\%}. Sub-indicator scores refer to average subject knowledge on a 0-100 scale.}
\end{table}

\hypertarget{table-4.-capacity-for-learning-jordan-2019} & & \cellcolor{yellow!15}\\\cdashline{1-2}
\hspace{1mm}\emph{Numeracy score}        & \cellcolor{green!15}92   & \multirow{-2}{*}{Nutrition Programs} & \multirow{-2}{*}{\cellcolor{yellow!15}3.5}  \\\cdashline{1-4}  
\hspace{1mm}\emph{Literacy score}        & \cellcolor{yellow!15}76   & Health Programs                     & \cellcolor{yellow!15}3.6 \\\cdashline{1-4}  
\hspace{1mm}\emph{Executive score}       & \cellcolor{yellow!15}68   & Center based care                   & \cellcolor{red!15}1.5 \\\cdashline{1-4}  
\hspace{1mm}\emph{Socio-emotional score} & \cellcolor{red!15}56   & Caregiver Skills Capacity           & \cellcolor{green!15}4.7 \\\cdashline{1-4}  
Student Attendance                     & {\cellcolor{green!15}97\%}     & Caregiver Financial Capacity        & \cellcolor{green!15}4.4 \\\hline
\end{tabular}}
\\
\setstretch{0.8}\color{darkgray}\scriptsize{\textit{Notes:} Capacity for learning indicates \% students with knowledge\textgreater{80\%}. Sub-indicator scores refer to average subject knowledge on a 0-100 scale.}
\end{table}

\hypertarget{table-5.-inputs-infrastructure-jordan-2019}{%
\paragraph{Table 5. Inputs \& infrastructure, Jordan,
2019}\label{table-5.-inputs-infrastructure-jordan-2019}}

\begin{table}[H]
\resizebox{\columnwidth}{!}{\begin{tabular}{m{3.8cm}cm{4.2cm}c}
\multicolumn{2}{c}{\textbf{Practice Indicators}} & \multicolumn{2}{c}{\textbf{Policy lever indicators}}\\\hline
Basic inputs & \cellcolor{green!15}4.3 & & \cellcolor{yellow!15} \\\cdashline{1-2}
\hspace{1mm}\emph{\% Blackboard}    & {\cellcolor{green!15}87\%} & & \cellcolor{yellow!15}\\\cdashline{1-2}
\hspace{1mm}\emph{\% Stationery}    & {\cellcolor{green!15}98\%} & & \cellcolor{yellow!15}\\\cdashline{1-2}
\hspace{1mm}\emph{\% Furniture}     & {\cellcolor{green!15}88\%} & & \cellcolor{yellow!15}\\\cdashline{1-2}
\hspace{1mm}\emph{\% EdTech access} & {\cellcolor{green!15}89\%} & \multirow{-5}{4.2cm}{Inputs and infrastructure standards} & \multirow{-5}{*}{\cellcolor{yellow!15}3.8}\\\hline
Basic infrastructure & \cellcolor{green!15}4.1 & & \cellcolor{yellow!15} \\\cdashline{1-2}
\hspace{1mm}\emph{\% Drinking water}    & {\cellcolor{green!15}89\%} & & \cellcolor{yellow!15}\\\cdashline{1-2}
\hspace{1mm}\emph{\% Functional toilet} & {\cellcolor{yellow!15}69\%} & & \cellcolor{yellow!15}\\\cdashline{1-2}
\hspace{1mm}\emph{\% Internet}          & {\cellcolor{green!15}100\%} & & \cellcolor{yellow!15}\\\cdashline{1-2}
\hspace{1mm}\emph{\% Electricity}       & {\cellcolor{green!15}88\%} & & \cellcolor{yellow!15}\\\cdashline{1-2}
\hspace{1mm}\emph{\% Disability access} & {\cellcolor{yellow!15}66\%} & \multirow{-6}{4.2cm}{Inputs and infrastructure monitoring} & \multirow{-6}{*}{\cellcolor{yellow!15}3}\\\hline
\end{tabular}}
\\
{\scriptsize
    \textcolor{darkgray}{\textit{Source:} \% refers to \% schools with the given sub-component}
  }

\end{table}

\vfill\null

\hypertarget{table-6.-school-management-by-principalsjordan-2019}{%
\paragraph{Table 6. School management by principalsJordan,
2019}\label{table-6.-school-management-by-principalsjordan-2019}}

\begin{table}[H]
\resizebox{\columnwidth}{!}{\begin{tabular}{m{4cm}cm{4cm}c}
\multicolumn{2}{c}{\textbf{Practice Indicators}} & \multicolumn{2}{c}{\textbf{Policy lever indicators}}\\\hline
Operational management                       & \cellcolor{green!15}4.7                  & \multirow{2}{*}{Clarity of functions} & \multirow{2}{*}{\cellcolor{green!15}5} \\\cdashline{1-2}
Instructional Leadership                     & \cellcolor{yellow!15}3.3                  & &  \cellcolor{green!15} \\\hline
\multirow{2}{*}{Principal School knowledge}  & \multirow{2}{*}{\cellcolor{yellow!15}3.8} & Attraction & \cellcolor{green!15}4.5\\\cdashline{3-4}
 & \cellcolor{yellow!15} & Selection \& Deployment & \cellcolor{green!15}4.6\\\cdashline{3-4}
\multirow{2}{*}{Principal Management skills} & \multirow{2}{*}{\cellcolor{green!15}4} & Support & \cellcolor{yellow!15}3.5\\\cdashline{3-4}
 & \cellcolor{green!15} & Evaluation              & \cellcolor{yellow!15}3.6\\\hline
\end{tabular}}
\\
{\scriptsize
    \textcolor{darkgray}{\textit{Source:} All scores are on a (0-5) scale.}
  }

\end{table}

\hypertarget{table-7.-politics-and-bureaucratic-capacity-indicators-jordan-2019}{%
\paragraph{Table 7. Politics and bureaucratic capacity indicators,
Jordan,
2019}\label{table-7.-politics-and-bureaucratic-capacity-indicators-jordan-2019}}

\begin{table}[H]
\resizebox{\columnwidth}{!}{\begin{tabular}{p{0.4\textwidth}c}
\textbf{Indicator} & \textbf{Value}\\ \hline
Quality of Bureaucracy & \cellcolor{yellow!15}3.6\\\cdashline{1-2}
Impartial Decision-Making & \cellcolor{yellow!15}3.1\\\cdashline{1-2}
Mandates \& Accountability & \cellcolor{green!15}4.2\\\cdashline{1-2}
National Learning Goals & \cellcolor{yellow!15}3.6\\\cdashline{1-2}
Financing & \cellcolor{green!15}4.7\\\hline
\end{tabular}}
\\
{\scriptsize
    \textcolor{darkgray}{\textit{Source:} All scores are on a (0-5) scale.}
  }

\end{table}
\raggedbottom

\hypertarget{gaps-between-de-facto-indicators-and-de-jure-poicies-in-teaching-and-school-management}{%
\subsubsection{\texorpdfstring{\textbf{GAPS BETWEEN DE-FACTO INDICATORS
AND DE-JURE POICIES IN TEACHING AND SCHOOL
MANAGEMENT}}{GAPS BETWEEN DE-FACTO INDICATORS AND DE-JURE POICIES IN TEACHING AND SCHOOL MANAGEMENT}}\label{gaps-between-de-facto-indicators-and-de-jure-poicies-in-teaching-and-school-management}}

A \textasciitilde{}0.8 point average gap exists between the de-facto
indicators and de-jure policies in Jordan across teaching and school
management. The highest gaps in practices and policies are observed for
Teaching - Support (2.4 points), Teaching- Intrinsic Motivation (1.6
points), Management-Evaluation (1.4 points).

\hypertarget{figure-4.-de-facto-and-de-jure-indicators-for-policy-levers-jordan}{%
\subsubsection{Figure 4. De-facto and de-jure indicators for policy
levers,
Jordan}\label{figure-4.-de-facto-and-de-jure-indicators-for-policy-levers-jordan}}

\includegraphics{C:/Users/wb469649/ONEDRI~1/DOCUME~1/Github/GEPD/COUNTR~1/Output/JORDAN~1/figure-latex/defacto_graph-1.pdf}

\end{document}
