% Options for packages loaded elsewhere
\PassOptionsToPackage{unicode}{hyperref}
\PassOptionsToPackage{hyphens}{url}
%
\documentclass[
  twocolumn]{article}
\usepackage{lmodern}
\usepackage{amssymb,amsmath}
\usepackage{ifxetex,ifluatex}
\ifnum 0\ifxetex 1\fi\ifluatex 1\fi=0 % if pdftex
  \usepackage[T1]{fontenc}
  \usepackage[utf8]{inputenc}
  \usepackage{textcomp} % provide euro and other symbols
\else % if luatex or xetex
  \usepackage{unicode-math}
  \defaultfontfeatures{Scale=MatchLowercase}
  \defaultfontfeatures[\rmfamily]{Ligatures=TeX,Scale=1}
\fi
% Use upquote if available, for straight quotes in verbatim environments
\IfFileExists{upquote.sty}{\usepackage{upquote}}{}
\IfFileExists{microtype.sty}{% use microtype if available
  \usepackage[]{microtype}
  \UseMicrotypeSet[protrusion]{basicmath} % disable protrusion for tt fonts
}{}
\makeatletter
\@ifundefined{KOMAClassName}{% if non-KOMA class
  \IfFileExists{parskip.sty}{%
    \usepackage{parskip}
  }{% else
    \setlength{\parindent}{0pt}
    \setlength{\parskip}{6pt plus 2pt minus 1pt}}
}{% if KOMA class
  \KOMAoptions{parskip=half}}
\makeatother
\usepackage{xcolor}
\IfFileExists{xurl.sty}{\usepackage{xurl}}{} % add URL line breaks if available
\IfFileExists{bookmark.sty}{\usepackage{bookmark}}{\usepackage{hyperref}}
\hypersetup{
  hidelinks,
  pdfcreator={LaTeX via pandoc}}
\urlstyle{same} % disable monospaced font for URLs
\usepackage[left=0.5cm,right=0.5cm,top=1.5cm,bottom=1.2cm]{geometry}
\usepackage{graphicx}
\makeatletter
\def\maxwidth{\ifdim\Gin@nat@width>\linewidth\linewidth\else\Gin@nat@width\fi}
\def\maxheight{\ifdim\Gin@nat@height>\textheight\textheight\else\Gin@nat@height\fi}
\makeatother
% Scale images if necessary, so that they will not overflow the page
% margins by default, and it is still possible to overwrite the defaults
% using explicit options in \includegraphics[width, height, ...]{}
\setkeys{Gin}{width=\maxwidth,height=\maxheight,keepaspectratio}
% Set default figure placement to htbp
\makeatletter
\def\fps@figure{htbp}
\makeatother
\setlength{\emergencystretch}{3em} % prevent overfull lines
\providecommand{\tightlist}{%
  \setlength{\itemsep}{0pt}\setlength{\parskip}{0pt}}
\setcounter{secnumdepth}{-\maxdimen} % remove section numbering
\usepackage{fancyhdr} \usepackage{booktabs,xcolor} \pagestyle{fancy} \renewcommand{\headrulewidth}{0pt} \rhead{2020} \cfoot{\includegraphics[width=20cm]{footer.png}} \fancypagestyle{plain}{\pagestyle{fancy}} \setlength{\headheight}{77.3pt} \setlength{\footskip}{35.1pt} \setlength{\textheight}{0.90\textheight} \pagenumbering{gobble} \usepackage[fontsize=9pt]{scrextend} \usepackage{float} \restylefloat{table} \usepackage{xcolor} \usepackage{multicol} \usepackage{array} \usepackage{colortbl} \usepackage{multirow} \usepackage{collcell} \usepackage{setspace} \usepackage{arydshln} \usepackage{caption} \captionsetup{skip=0pt} \setlength{\columnsep}{1.5cm} \setstretch{1}
\ifluatex
  \usepackage{selnolig}  % disable illegal ligatures
\fi

\author{}
\date{\vspace{-2.5em}}

\begin{document}

\newcommand{\greynote}[1]{
    {\scriptsize
    \textcolor{darkgray}{\textit{Notes:} #1}
  }
}

\newcommand{\greysource}[1]{
    {\scriptsize
    \textcolor{darkgray}{\textit{Source:} #1}
  }
}

\newcommand{\greydisclaimer}[1]{
    {\scriptsize
    \textcolor{darkgray}{\textit{Disclaimer:} #1}
  }
}

\newcommand{\greytext}[1]{
    {\scriptsize
    \textcolor{darkgray}{#1}
  }
}

\newcommand*{\tabindent}{\hspace{1mm}}

\hypertarget{introduction}{%
\subsubsection{\texorpdfstring{\textbf{INTRODUCTION}}{INTRODUCTION}}\label{introduction}}

\textbf{The Global Education Policy Dashboard (GEPD): An innovative tool
to measure drivers of learning outcomes in basic education}\\
GEPD uses 3 data collection instruments to report on nearly 40
indicators that operationalize the World Development Report 2018
framework to track 3 areas for progress in education- Practices,
Policies, and Politics. Using these indicators, the dashboard highlights
areas where countries need to act to improve learning outcomes and
allows a way for governments to track progress as they act to close gaps
in these areas. For more information on GEPD, please visit
\textbf{\href{https://www.worldbank.org/en/topic/education/brief/global-education-policy-dashboard}{www.worldbank.org/global-education-policy-dashboard}}

\hypertarget{figure-1.-gepd-framework-practices-policies-and-politics-expanding-on-wdr-2018-framework}{%
\paragraph{Figure 1. GEPD framework (practices, policies and politics),
expanding on WDR 2018
framework}\label{figure-1.-gepd-framework-practices-policies-and-politics-expanding-on-wdr-2018-framework}}

\begin{center}\includegraphics[width=0.8\linewidth]{/Users/kanikaverma/Documents/GitHub/gepd/Country_Reports/Data/full_circle} \end{center}

\hypertarget{instruments-of-data}{%
\subsubsection{\texorpdfstring{\textbf{INSTRUMENTS OF
DATA}}{INSTRUMENTS OF DATA}}\label{instruments-of-data}}

The \textbf{School Survey} consists of 8 modules to collect data across
200-300 schools on practices (the quality of service delivery in
schools) and de facto policy indicators. It consists of streamlined
versions of existing instruments together with new questions to fill
gaps in those instruments.\\
The \textbf{Policy Survey} collects information via interviews with
\textasciitilde200 officials per country at federal and regional level
to feed into the policy de jure indicators and identify key elements of
the policy framework.\\
The \textbf{Survey of public officials} collects information about the
capacity and orientation of the bureaucracy and political factors
affecting education outcomes. This survey is an education-focused
version of the civil-servant surveys from the Bureaucracy Lab, WBG.

\hypertarget{key-takeaways-2020}{%
\subsubsection{\texorpdfstring{\textbf{KEY TAKEAWAYS,
\uppercase{Rwanda},
2020}}{KEY TAKEAWAYS, , 2020}}\label{key-takeaways-2020}}

\begin{itemize}
\tightlist
\item
  Substantially low learning outcomes observed for students in Grade 1
  and 4.
\item
  Weak teacher pedagogical skills and low teacher content knowledge
  attributed to poor teaching support and weak monitoring and
  accountability systems.
\item
  Grade 1 proficiency of students is \textasciitilde9\%, with students
  scoring lower on executive functions and socio-emotional learning.
  Only 13\% students are enrolled in early childhood programs which face
  gaps in caregiver skills and financial constraints.
\item
  Basic inputs and infrastructure are weak in areas of functional
  toilets and electricity in schools.
\item
  Major gaps are seen in implementation of teaching support policies,
  teaching monitoring and accountability systems and selection and
  deployment policies for school principals.
\item
  Primary education funding amount and efficiency of spending is low and
  education policy implementation is politicized, lowering bureaucratic
  capacity.
\end{itemize}

\setlength\dashlinedash{0.2pt}
\setlength\dashlinegap{1.5pt}
\setlength\arrayrulewidth{0.3pt}
\definecolor{outcome}{RGB}{231, 128, 65}
\definecolor{practice}{RGB}{69, 164, 211}
\definecolor{policy}{RGB}{37, 117, 186}
\definecolor{politics}{RGB}{0, 35, 69}

\hypertarget{table-1.-gepd-indicators-rwanda-2019}{%
\paragraph{Table 1. GEPD Indicators, Rwanda,
2019}\label{table-1.-gepd-indicators-rwanda-2019}}

\begin{table}[H]
\resizebox{\columnwidth}{!}{\begin{tabular}{cm{2cm}m{5cm}c}
& \textbf{Subtitle} & \textbf{Indicator} & \textbf{Overall} \\\hline
\cellcolor{outcome} & \multirow{3}{2cm}{Learning} & Proficiency by Grade 2/3       & {NA} \\\cdashline{3-4}
\cellcolor{outcome} &                             & Proficiency by End of Primary  & {NA} \\\cdashline{3-4}
\cellcolor{outcome} &                             & Proficiency on GEPD Assessment & {\cellcolor{red!15}0.2\%} \\\cdashline{2-4}
\cellcolor{outcome} &  Participation              & Net Adjusted Enrollment Rate   & {\cellcolor{green!15}95\%} \\\cdashline{2-4}
\cellcolor{outcome}\multirow{-5}{*}{\rotatebox{90}{\textcolor{white}{Outcomes}}} &  Participation and learning  & Learning adjusted years in schooling (in years) & {\cellcolor{red!15}3.9}\\\cdashline{1-4}
\cellcolor{practice} & \multirow{3}{2cm}{Teaching}                 & Teacher Effort           & {\cellcolor{red!15}80\%} \\\cdashline{3-4}
\cellcolor{practice} &                                             & Content Knowledge        & {\cellcolor{red!15}27\%} \\\cdashline{3-4}
\cellcolor{practice} &                                             & Pedagogical Skills       & {\cellcolor{red!15}22\%} \\\cdashline{2-4}
\cellcolor{practice} & \multirow{2}{2cm}{Inputs \& Infrastructure} & Basic Inputs             & \cellcolor{yellow!15}3.3 \\\cdashline{3-4}
\cellcolor{practice} &                                             & Basic Infrastructure     & \cellcolor{yellow!15}3.1 \\\cdashline{2-4}
\cellcolor{practice} & \multirow{2}{2cm}{Learners}                 & Capacity for Learning    & {\cellcolor{red!15}9\%} \\\cdashline{3-4}
\cellcolor{practice} &                                             & Student Attendance       & {\cellcolor{yellow!15}87\%} \\\cdashline{2-4}
\cellcolor{practice} & \multirow{4}{2cm}{School Management}        & Operational Management   & \cellcolor{green!15}4 \\\cdashline{3-4}
\cellcolor{practice} &                                             & Instructional Leadership & \cellcolor{yellow!15}3.4 \\\cdashline{3-4}
\cellcolor{practice} &                                             & School Knowledge         & \cellcolor{green!15}4 \\\cdashline{3-4}
\cellcolor{practice}\multirow{-13}{*}{\rotatebox{90}{\textcolor{white}{Practices}}}  & & Management Practices  & \cellcolor{green!15}4.2 \\\cdashline{1-4}
\cellcolor{policy}   & \multirow{6}{2cm}{Teaching}                  & Attraction                    & \cellcolor{yellow!15}3.9 \\\cdashline{3-4}
\cellcolor{policy}   &                                              & Selection \& Deployment       & \cellcolor{yellow!15}3.3 \\\cdashline{3-4}
\cellcolor{policy}   &                                              & Support                       & \cellcolor{red!15}2.8 \\\cdashline{3-4}
\cellcolor{policy}   &                                              & Evaluation                    & \cellcolor{green!15}4.5 \\\cdashline{3-4}
\cellcolor{policy}   &                                              & Monitoring \& Accountability  & \cellcolor{red!15}2.9 \\\cdashline{3-4}
\cellcolor{policy}   &                                              & Intrinsic Motivation          & \cellcolor{yellow!15}3.9 \\\cdashline{2-4}
\cellcolor{policy}   & \multirow{2}{2cm}{Inputs \& infrastructure}  & Standards                     & \cellcolor{green!15}4.5 \\\cdashline{3-4} 
\cellcolor{policy}   &                                              & Monitoring                    & \cellcolor{yellow!15}3.1 \\\cdashline{2-4}
\cellcolor{policy}   & \multirow{5}{2cm}{Learners}                  & Nutrition Programs            & \cellcolor{yellow!15}3.9 \\\cdashline{3-4}
\cellcolor{policy}   &                                              & Health Programs               & \cellcolor{yellow!15}3.9 \\\cdashline{3-4}
\cellcolor{policy}   &                                              & Center based care             & \cellcolor{red!15}1.5 \\\cdashline{3-4}
\cellcolor{policy}   &                                              & Caregiver Financial Capacity  & \cellcolor{red!15}1.5 \\\cdashline{3-4}
\cellcolor{policy}   &                                              & Caregiver Skills Capacity     & \cellcolor{red!15}2.8 \\\cdashline{2-4}
\cellcolor{policy}   & \multirow{5}{2cm}{School Management}         & Clarity of Functions          & \cellcolor{green!15}4.9 \\\cdashline{3-4}
\cellcolor{policy}   &                                              & Attraction                    & \cellcolor{green!15}4.3 \\\cdashline{3-4}
\cellcolor{policy}   &                                              & Selection \& Deployment       & \cellcolor{yellow!15}3.9 \\\cdashline{3-4}
\cellcolor{policy}   &                                              & Support                       & \cellcolor{yellow!15}3.7 \\\cdashline{3-4}
\cellcolor{policy}\multirow{-18}{*}{\rotatebox{90}{\textcolor{white}{Policy levers}}} & & Evaluation           & \cellcolor{green!15}4.5 \\\cdashline{1-4}
\cellcolor{politics} & \multirow{5}{2cm}{Politics \& Bureaucratic Capacity}           & Quality of Bureaucracy & \cellcolor{yellow!15}3.9 \\\cdashline{3-4}
\cellcolor{politics} & & Impartial Decision-Making  & \cellcolor{yellow!15}3.3 \\\cdashline{3-4}
\cellcolor{politics} & & Mandates \& Accountability & \cellcolor{red!15}2.8 \\\cdashline{3-4}
\cellcolor{politics} & & National Learning Goals    & \cellcolor{green!15}4.5 \\\cdashline{3-4}
\cellcolor{politics}\multirow{-5}{*}{\rotatebox{90}{\textcolor{white}{Politics}}}     & & Financing            & \cellcolor{red!15}2.9 \\\hline
\end{tabular}}
\\
\setstretch{0.8}\color{darkgray}\scriptsize{\textit{Source:} UIS, GLAD, GEPD, World Bank, Rwanda, 2020. For information on indicators, please consult the World Bank \href{https://github.com/worldbank/GEPD}{\underline{GEPD}}, \href{https://github.com/worldbank/GLAD}{\underline{GLAD}} and \href{https://github.com/worldbank/LearningPoverty}{\underline{Learning Poverty}} repositories.}\\
\setstretch{0.8}\color{darkgray}\scriptsize{\textit{Notes:} (1) Proficiency on GEPD assessment means \% students with knowledge\textgreater{90\%}. (2) Proficiency by end of primary uses threshold as per Minimum Proficiency Levels set by GAML(UIS). (3) All indicators are on a scale of 0-5 unless measured in \%. (4) Green indicates indicator 'on-target', yellow indicates 'requires caution', red indicates 'needs improvement'.}
\end{table}

\hypertarget{table-2.-gepd-indicators-by-gender-rwanda-2020} & {\cellcolor{red!15}0.2\%} \\\cdashline{1-4}
\multirow{3}{*}{Teaching} & Teacher Effort      & {\cellcolor{red!15}79\%} & {\cellcolor{red!15}80\%} \\\cdashline{2-4}
                          &  Content knowledge  & {\cellcolor{red!15}28\%} & {\cellcolor{red!15}26\%} \\\cdashline{2-4}
                          &  Pedagogical Skills & {NA} & {NA} \\\cdashline{1-4}
\multirow{2}{*}{Learners}   & Capacity for Learning    & {\cellcolor{red!15}52\%} & {\cellcolor{red!15}55\%} \\\cdashline{2-4}
                            & Student Attendance       & {\cellcolor{red!15}82\%} & {\cellcolor{red!15}84\%} \\\cdashline{1-4}
\multirow{4}{2cm}{School Management}  & Operational Management   & \cellcolor{green!15}4 & \cellcolor{green!15}4 \\\cdashline{2-4}
                                      & Instructional Leadership & \cellcolor{yellow!15}3.4 & \cellcolor{yellow!15}3.4 \\\cdashline{2-4}
                                      & School Knowledge         & \cellcolor{green!15}4 & \cellcolor{green!15}4 \\\cdashline{2-4}
                                      & Management Practices     & \cellcolor{green!15}4.2 & \cellcolor{green!15}4.1\\\hline
\end{tabular}}
\end{table}
\raggedbottom

\hypertarget{table-3.-gepd-indicators-by-region-rwanda-2020} & {\cellcolor{red!15}0.8\%} \\\cdashline{1-4}
\multirow{3}{*}{Teaching} & Teacher Effort                 & {\cellcolor{red!15}79\%} & {\cellcolor{yellow!15}85\%} \\\cdashline{2-4}
                          &  Content knowledge             & {\cellcolor{red!15}24\%} & {\cellcolor{red!15}45\%} \\\cdashline{2-4}
                          &  Pedagogical Skills            & {\cellcolor{red!15}23\%} & {\cellcolor{red!15}13\%} \\\cdashline{1-4}
\multirow{2}{2cm}{Inputs \& infrastructure}   & Basic Inputs             & \cellcolor{yellow!15}3.2 & \cellcolor{yellow!15}3.8 \\\cdashline{2-4}
                                              & Basic Infrastructure     & \cellcolor{yellow!15}3.1 & \cellcolor{yellow!15}3.5 \\\cdashline{1-4}
\multirow{2}{*}{Learners}   & Capacity for Learning    & {\cellcolor{red!15}52\%} & {\cellcolor{red!15}63\%} \\\cdashline{2-4}
                            & Student Attendance       & {\cellcolor{yellow!15}86\%} & {\cellcolor{green!15}93\%} \\\cdashline{1-4}
\multirow{4}{2cm}{School Management}  & Operational Management   & \cellcolor{green!15}4 & \cellcolor{green!15}4.1 \\\cdashline{2-4}
                                      & Instructional Leadership & \cellcolor{yellow!15}3.4 & \cellcolor{yellow!15}3.4 \\\cdashline{2-4}
                                      & School Knowledge         & \cellcolor{yellow!15}3.9 & \cellcolor{green!15}4.5 \\\cdashline{2-4}
                                      & Management Practices     & \cellcolor{green!15}4.2 & \cellcolor{green!15}4\\\hline
\end{tabular}}
\end{table}
\raggedbottom

\hypertarget{learning-outcomes-3.9-learning-adjusted-years-in-school-0.2-gepd-proficiency-in-grade-4}{%
\subsubsection{\texorpdfstring{\textbf{LEARNING OUTCOMES: 3.9 LEARNING
ADJUSTED YEARS IN SCHOOL, 0.2\% GEPD PROFICIENCY IN GRADE
4}}{LEARNING OUTCOMES: 3.9 LEARNING ADJUSTED YEARS IN SCHOOL, 0.2\% GEPD PROFICIENCY IN GRADE 4}}\label{learning-outcomes-3.9-learning-adjusted-years-in-school-0.2-gepd-proficiency-in-grade-4}}

Learning adjusted years of school (LAYS) is calculated by adjusting
expected years of schooling for schooling quality. Learning adjusted
years of schooling in Rwanda is 0.9 years lower than the average for
Sub-Saharan Africa (excluding high income) region and 0.5 years lower
than the average for Low income countries.

\hypertarget{figure-2.-learning-adjusted-years-in-school-comparison} means 0.2\% students score
greater than 90\% in GEPD assessment. Student proficiency is 1 points
higher in language compared to numeracy, 0.1 points lower for boys
compared to girls, and 0.7 points higher in urban areas compared to
rural areas. \vfill\null

\hypertarget{figure-3.-grade-4-proficiency-rwanda}{%
\paragraph{Figure 3. Grade 4 proficiency,
Rwanda}\label{figure-3.-grade-4-proficiency-rwanda}}

\includegraphics{/Users/kanikaverma/Documents/GitHub/gepd/Country_Reports/Output/Rwanda_2019_4Pager_files/figure-latex/lern_teach-1.pdf}

\hypertarget{comparing-de-facto-practices-and-policy-levers}{%
\subsubsection{\texorpdfstring{\textbf{COMPARING DE-FACTO PRACTICES AND
POLICY
LEVERS}}{COMPARING DE-FACTO PRACTICES AND POLICY LEVERS}}\label{comparing-de-facto-practices-and-policy-levers}}

Practice indicators measure quality of service delivery in schools such
as teacher and student attendance, teacher knowledge, principal
management skills, etc. Policy lever indicators measure how well school,
personnel and student policies governing these practices are
implemented. For instance, teacher content knowledge(practice) is
influenced by implementation of teacher hiring policies, teacher
training and monitoring and accountability systems. Comparing de-facto
practice and policy lever indicators allows identification of
low-scoring policy levers that affect observed practice indicators.

\hypertarget{teacher-effectiveness}{%
\paragraph{\texorpdfstring{\textbf{Teacher
effectiveness}}{Teacher effectiveness}}\label{teacher-effectiveness}}

Teacher effectiveness is determined by a teacher's own knowledge level,
pedagogical skills and effort. These practices are impacted by quality
of teaching support, teacher recruitment and deployment, level of
teaching attraction (incentives/job satisfaction), teacher motivation
and monitoring systems. Teacher content knowledge (27\%) needs
improvement. Teacher proficiency in language (20\%) is 21 points lower
than mathematics proficiency (41\%). Teacher pedagogical skills score
(22\%) needs improvement, and teacher attendance (87\%) requires
caution. Teaching - Support is the weakest policy lever(2.8/5).

\begin{table}[H]
\resizebox{\columnwidth}{!}{\begin{tabular}{m{3.8cm}cm{4.2cm}c}
\multicolumn{2}{c}{\textbf{Practice Indicators}} & \multicolumn{2}{c}{\textbf{Policy levers(Teaching)}}\\\hline
Content knowledge                   & {\cellcolor{red!15}27\%}   & Attraction                                & \cellcolor{yellow!15}3.9 \\\cdashline{1-4} 
\hspace{1mm}\emph{Maths proficiency}               & {\cellcolor{red!15}41\%} &                                           & \cellcolor{yellow!15}\\\cdashline{1-2}   
\hspace{1mm}\emph{Language proficiency}            & {\cellcolor{red!15}20\%} & \multirow{-2}{*}{Selection \& deployment} & \multirow{-2}{*}{\cellcolor{yellow!15}3.3} \\\cdashline{1-4}        
Pedagogical skills                               & {\cellcolor{red!15}22\%}   &                                           & \cellcolor{red!15}\\\cdashline{1-2}
\hspace{1mm}\emph{\% Classroom culture}            & {\cellcolor{green!15}90\%} & \multirow{-2}{*}{Support}                 & \multirow{-2}{*}{\cellcolor{red!15}2.8} \\\cdashline{1-4}
\hspace{1mm}\emph{\% Instruction practices}        & {\cellcolor{red!15}20\%} &                                           & \cellcolor{green!15}\\\cdashline{1-2}
\hspace{1mm}\emph{\% Socio-emotional skills}       & {\cellcolor{red!15}15\%} & \multirow{-2}{*}{Evaluation}              & \multirow{-2}{*}{\cellcolor{green!15}4.5} \\\cdashline{1-4}
& \cellcolor{yellow!15} & Monitoring \& Accountability  & \cellcolor{red!15}2.9 \\\cdashline{3-4}
\multirow{-2}{*}{Teacher Attendance}             & \multirow{-2}{*}{\cellcolor{yellow!15}87\%} & Intrinsic motivation    & \cellcolor{yellow!15}3.9 \\\hline
\end{tabular}}
\\
\setstretch{0.8}\color{darkgray}\scriptsize{\textit{Notes:} Content knowledge(\& sub-indicators) indicate \% teachers with knowledge\textgreater{90\%}.Pedagogical skills(\& sub-indicators) indicate \% teachers with proficiency 3/5 or above.}
\end{table}
\raggedbottom

\hypertarget{capacity-for-learning-in-grade-1}{%
\paragraph{\texorpdfstring{\textbf{Capacity for learning in Grade
1}}{Capacity for learning in Grade 1}}\label{capacity-for-learning-in-grade-1}}

Early learning is affected by quality of implementation of health and
nutrition programs, enrolment in early childhood education, quality of
skills of educators and financial support provided to programs enabling
early learning. Proficiency in Grade 1 (9\%) needs improvement. Literacy
score(38)\% is the lowest knowledge sub-score. Student attendance(87\%)
requires caution. Center-Based Care is the weakest policy lever(1.5/5).

\begin{table}[H]
\resizebox{\columnwidth}{!}{\begin{tabular}{m{3.8cm}cm{4.2cm}c}
\multicolumn{2}{c}{\textbf{Practice Indicators}} & \multicolumn{2}{c}{\textbf{Policy levers(Learners)}}\\\hline
Capacity for learning                  & {\cellcolor{red!15}9\%} & & \cellcolor{yellow!15}\\\cdashline{1-2}
\hspace{1mm}\emph{Numeracy score}        & \cellcolor{red!15}62   & \multirow{-2}{*}{Nutrition Programs} & \multirow{-2}{*}{\cellcolor{yellow!15}3.2}  \\\cdashline{1-4}  
\hspace{1mm}\emph{Literacy score}        & \cellcolor{red!15}38   & Health Programs                     & \cellcolor{yellow!15}3.5 \\\cdashline{1-4}  
\hspace{1mm}\emph{Executive score}       & \cellcolor{red!15}47   & Center based care                   & \cellcolor{red!15}1.5 \\\cdashline{1-4}  
\hspace{1mm}\emph{Socio-emotional score} & \cellcolor{red!15}67   & Caregiver Skills Capacity           & \cellcolor{red!15}2.8 \\\cdashline{1-4}  
Student Attendance                     & {\cellcolor{yellow!15}87\%}     & Caregiver Financial Capacity        & \cellcolor{red!15}2.3 \\\hline
\end{tabular}}
\\
\setstretch{0.8}\color{darkgray}\scriptsize{\textit{Notes:} Capacity for learning indicates \% students with knowledge\textgreater{90\%}. Sub-indicator scores refer to average subject knowledge on a 0-100 scale.}
\end{table}
\raggedbottom

\hypertarget{inputs-infrastructure}{%
\paragraph{\texorpdfstring{\textbf{Inputs \&
Infrastructure}}{Inputs \& Infrastructure}}\label{inputs-infrastructure}}

Quality of school inputs and infrastrucutre is affected by physical
infrastructure standards set in policies, and strength of school
monitoring systems. Basic inputs (3.3/5) are on target. Percent of
classrooms with a functional blackboard and chalk(60)\% is the lowest
score. Basic infrastructure (3.1/5) requires caution. Percent of schools
with access to internet(27)\% is the lowest score. Inputs \&
Infrastructure - Monitoring is the weakest policy lever(3.1/5).

\begin{table}[H]
\resizebox{\columnwidth}{!}{\begin{tabular}{m{3.8cm}cm{4.2cm}c}
\multicolumn{2}{c}{\textbf{Practice Indicators}} & \multicolumn{2}{c}{\textbf{Policy levers(Inputs)}}\\\hline
Basic inputs & \cellcolor{yellow!15}3.3 & & \cellcolor{green!15} \\\cdashline{1-2}
\hspace{1mm}\emph{\% Blackboard}    & {\cellcolor{red!15}60\%} & & \cellcolor{green!15}\\\cdashline{1-2}
\hspace{1mm}\emph{\% Stationery}    & {\cellcolor{green!15}92\%} & & \cellcolor{green!15}\\\cdashline{1-2}
\hspace{1mm}\emph{\% Furniture}     & {\cellcolor{green!15}99\%} & & \cellcolor{green!15}\\\cdashline{1-2}
\hspace{1mm}\emph{\% EdTech access} & {\cellcolor{red!15}76\%} & \multirow{-5}{4.2cm}{Inputs and infrastructure standards} & \multirow{-5}{*}{\cellcolor{green!15}4.5}\\\hline
Basic infrastructure & \cellcolor{yellow!15}3.1 & & \cellcolor{yellow!15} \\\cdashline{1-2}
\hspace{1mm}\emph{\% Drinking water}    & {\cellcolor{red!15}74\%} & & \cellcolor{yellow!15}\\\cdashline{1-2}
\hspace{1mm}\emph{\% Functional toilet} & {\cellcolor{red!15}57\%} & & \cellcolor{yellow!15}\\\cdashline{1-2}
\hspace{1mm}\emph{\% Internet}          & {\cellcolor{red!15}78\%} & & \cellcolor{yellow!15}\\\cdashline{1-2}
\hspace{1mm}\emph{\% Electricity}       & {\cellcolor{red!15}27\%} & & \cellcolor{yellow!15}\\\cdashline{1-2}
\hspace{1mm}\emph{\% Disability access} & {\cellcolor{red!15}75\%} & \multirow{-6}{4.2cm}{Inputs and infrastructure monitoring} & \multirow{-6}{*}{\cellcolor{yellow!15}3.1}\\\hline
\end{tabular}}
\\
\setstretch{0.8}\color{darkgray}\scriptsize{\textit{Notes:} \% refers to \% schools with the given sub-component}
\end{table}
\raggedbottom

\hypertarget{school-management-by-principals}{%
\paragraph{\texorpdfstring{\textbf{School Management by
principals}}{School Management by principals}}\label{school-management-by-principals}}

School management practices of principals are impacted by clarity in
assignment of responsibilities and quality of support systems for school
leaders, principal recruitment and deployment, incentives and evaluation
systems. In school management, the lowest score is for principal's
Instructional Leadership(3.4/5), whereas the highest score is obtained
for Principal Management Skills(4.2/5). School Management- Support is
the weakest policy lever(3.7/5).

\begin{table}[H]
\resizebox{\columnwidth}{!}{\begin{tabular}{m{4cm}cm{3.3cm}c}
\multicolumn{2}{c}{\textbf{Practice Indicators}} & \multicolumn{2}{c}{\textbf{Policy levers(Management)}}\\\hline
Operational management                     & \cellcolor{green!15}4       & & \cellcolor{green!15}\\\cdashline{1-2}
\hspace{1mm}\emph{Infrastructure}            & \cellcolor{green!15}4.5     & & \cellcolor{green!15}\\\cdashline{1-2}
\hspace{1mm}\emph{Ensuring inputs}           & \cellcolor{yellow!15}3.5     & & \cellcolor{green!15}\\\cdashline{1-2}
Instructional Leadership                   & \cellcolor{yellow!15}3.4       & & \cellcolor{green!15}\\\cdashline{1-2}
\hspace{1mm}\emph{\% Classroom observed}     & {\cellcolor{yellow!15}88\%} & & \cellcolor{green!15}\\\cdashline{1-2}
\hspace{1mm}\emph{\% Discussed observations} & {\cellcolor{red!15}51\%} & & \cellcolor{green!15}\\\cdashline{1-2}
\hspace{1mm}\emph{\% Feedback given}         & {\cellcolor{red!15}70\%} & & \cellcolor{green!15}\\\cdashline{1-2}
\hspace{1mm}\emph{\% Lesson-plan feedback}   & {\cellcolor{red!15}44\%} & \multirow{-8}{*}{Clarity of functions} & \multirow{-8}{*}{\cellcolor{green!15}4.9} \\\hline
Principal school knowledge                         & \cellcolor{green!15}4       & & \cellcolor{green!15}\\\cdashline{1-2}
\hspace{1mm}\emph{\% Teachers' knowledge}  & {\cellcolor{red!15}80\%} & \multirow{-2}{4cm}{Attraction} & \multirow{-2}{*}{\cellcolor{green!15}4.3}\\\cdashline{1-4}
\hspace{1mm}\emph{\% Teachers' experience} & {\cellcolor{green!15}99\%} & & \cellcolor{yellow!15}\\\cdashline{1-2}
\hspace{1mm}\emph{\% Input availability}      & {\cellcolor{red!15}77\%} & \multirow{-2}{4cm}{Selection \& Deployment} & \multirow{-2}{*}{\cellcolor{yellow!15}3.9} \\\cdashline{1-4} 
Principal Management skills & \cellcolor{green!15}4.2 & & \cellcolor{yellow!15}\\\cdashline{1-2}
\hspace{1mm}\emph{Problem solving score} & \cellcolor{green!15}4.2 & \multirow{-2}{4cm}{Support} & \multirow{-2}{*}{\cellcolor{yellow!15}3.7}\\\cdashline{1-4}
\hspace{1mm}\emph{Goal-setting score} & \cellcolor{green!15}4.2 & Evaluation & \cellcolor{green!15}4.5\\\hline
\end{tabular}}
\\
\setstretch{0.8}\color{darkgray}\scriptsize{\textit{Notes:} (1) Under instructional leadership, \% refers to \% teachers reporting in affirmative for the given sub-component. (2) Under principal school knowledge, \% refers to \% principals familiar with the given sub-component in the school.}
\end{table}
\raggedbottom

\hypertarget{gaps-between-de-facto-policy-levers-and-de-jure-policies-in-teaching-and-school-management}{%
\subsubsection{\texorpdfstring{\textbf{GAPS BETWEEN DE-FACTO POLICY
LEVERS AND DE-JURE POLICIES IN TEACHING AND SCHOOL
MANAGEMENT}}{GAPS BETWEEN DE-FACTO POLICY LEVERS AND DE-JURE POLICIES IN TEACHING AND SCHOOL MANAGEMENT}}\label{gaps-between-de-facto-policy-levers-and-de-jure-policies-in-teaching-and-school-management}}

De-facto policy levers measure how well school, teacher and student
policies are being implemented in the school system. GEPD also measures
de-jure policy indicators which measure the strength and quality of the
underlying student, teacher and school management policies. Analysis of
the difference in these indicators shows gaps in implementation of
education policies in schools.

A \textasciitilde{}\textbf{0.7 point average gap} exists between the
de-facto policy levers and de-jure policies in Rwanda across teaching
and school management, suggesting there are some gaps in policy
implementation in teaching and school management. Smallest gaps
suggesting good level of implementation are observed for Teaching -
Monitoring \& Accountability(-1.9 points), Management-Clarity of
functions(0.1 points) and Teaching- Evaluation(0.5 points).

\hypertarget{figure-4.-de-facto-and-de-jure-indicators-for-policy-levers-rwanda}{%
\paragraph{Figure 4. De-facto and de-jure indicators for policy levers,
Rwanda}\label{figure-4.-de-facto-and-de-jure-indicators-for-policy-levers-rwanda}}

\includegraphics{/Users/kanikaverma/Documents/GitHub/gepd/Country_Reports/Output/Rwanda_2019_4Pager_files/figure-latex/defacto_graphs-1.pdf}

\hypertarget{areas-in-teaching-and-school-management-with-highest-gaps-in-de-facto-policy-levers-and-de-jure-policies}{%
\subsubsection{\texorpdfstring{\textbf{AREAS IN TEACHING AND SCHOOL
MANAGEMENT WITH HIGHEST GAPS IN DE-FACTO POLICY LEVERS AND DE-JURE
POLICIES}}{AREAS IN TEACHING AND SCHOOL MANAGEMENT WITH HIGHEST GAPS IN DE-FACTO POLICY LEVERS AND DE-JURE POLICIES}}\label{areas-in-teaching-and-school-management-with-highest-gaps-in-de-facto-policy-levers-and-de-jure-policies}}

Largest gaps indicating a mismatch in policy design and policy
implementation in schools are observed for Teaching - Support(2.2
points), Teaching - Selection \& Deployment(1.4 points), Management-
Selection \& Deployment(1.1 points). The breakdown of sub-indicators
within these de-facto policy levers show the specific areas where scores
are the lowest, contributing to the gap observed in policy
implementation. \vfill\null

\hypertarget{figure-5-sub-indicators-of-teaching---support}{%
\paragraph{Figure 5: Sub-indicators of Teaching -
Support}\label{figure-5-sub-indicators-of-teaching---support}}

\includegraphics{/Users/kanikaverma/Documents/GitHub/gepd/Country_Reports/Output/Rwanda_2019_4Pager_files/figure-latex/defacto_sub_graphs_1-1.pdf}

{\scriptsize
    \textcolor{darkgray}{\textit{Notes:} Percent of teachers reporting in affirmative unless stated otherwise}
  }

\hypertarget{figure-6-sub-indicators-of-teaching---selection-deployment}{%
\paragraph{Figure 6: Sub-indicators of Teaching - Selection \&
Deployment}\label{figure-6-sub-indicators-of-teaching---selection-deployment}}

\includegraphics{/Users/kanikaverma/Documents/GitHub/gepd/Country_Reports/Output/Rwanda_2019_4Pager_files/figure-latex/defacto_sub_graphs_2-1.pdf}

{\scriptsize
    \textcolor{darkgray}{\textit{Notes:} Average quality of applicants is based on benchmarking secondary exit exams performance of teachers.}
  }

\hypertarget{figure-7-sub-indicators-of-management--selection-deployment}{%
\paragraph{Figure 7: Sub-indicators of Management- Selection \&
Deployment}\label{figure-7-sub-indicators-of-management--selection-deployment}}

\includegraphics{/Users/kanikaverma/Documents/GitHub/gepd/Country_Reports/Output/Rwanda_2019_4Pager_files/figure-latex/defacto_sub_graphs_3-1.pdf}

{\scriptsize
    \textcolor{darkgray}{\textit{Notes:} Percent of principals reporting which factors count in principal selection, or are most important in principal selection.}
  }

\vfill\null

\hypertarget{politics-bureaucratic-capacity-indicators}{%
\subsubsection{\texorpdfstring{\textbf{POLITICS \& BUREAUCRATIC CAPACITY
INDICATORS}}{POLITICS \& BUREAUCRATIC CAPACITY INDICATORS}}\label{politics-bureaucratic-capacity-indicators}}

Politics and bureaucratic capacity indicators measure the capacity and
orientation of the bureaucracy, as well as political factors affecting
education outcomes. The highest score in politics and bureaucratic
capacity is noted for Mandates \& Accountability (4.5/5), and the lowest
score is noted for Financing (2.4/5).

\hypertarget{figure-8.-politics-and-bureaucratic-capacity-indicators-rwanda}{%
\paragraph{Figure 8. Politics and bureaucratic capacity indicators,
Rwanda}\label{figure-8.-politics-and-bureaucratic-capacity-indicators-rwanda}}

\includegraphics{/Users/kanikaverma/Documents/GitHub/gepd/Country_Reports/Output/Rwanda_2019_4Pager_files/figure-latex/politics_graph-1.pdf}

\hypertarget{table-4.-politics-and-bureaucratic-capacity-sub-indicators-rwanda}{%
\subsubsection{Table 4. Politics and bureaucratic capacity
sub-indicators,
Rwanda}\label{table-4.-politics-and-bureaucratic-capacity-sub-indicators-rwanda}}

\begin{table}[H]
\resizebox{\columnwidth}{!}{\begin{tabular}{m{2cm}m{6cm}c}
\textbf{Indicator} & \textbf{Sub-indicator}  & \textbf{Value}  \\\hline
                                   & Knowledge and skills & \cellcolor{green!15}4.9 \\\cdashline{2-3}
                                   & Work enviornment &\cellcolor{green!15}4.3 \\\cdashline{2-3}
                                   & Merit &\cellcolor{green!15}4.2 \\\cdashline{2-3}
\multirow{-4}{2cm}{Quality of bureaucracy (Scale 1-5)} & Motivation and attitudes &\cellcolor{green!15}4.4 \\\hline
                                   & Politicized personnel management &\cellcolor{yellow!15}3.8 \\\cdashline{2-3}
                                   & Politicized policy-making &\cellcolor{yellow!15}3.8 \\\cdashline{2-3}
                                   & Politicized policy implementation &\cellcolor{yellow!15}3.1 \\\cdashline{2-3}
\multirow{-4}{2cm}{Impartial decision making (Scale 1-5)} & Employee unions as facilitators &\cellcolor{green!15}4.7 \\\hline
                                   & Coherence &\cellcolor{green!15}4.1 \\\cdashline{2-3}
                                   & Transparency &\cellcolor{green!15}4.5 \\\cdashline{2-3}
\multirow{-3}{2cm}{Mandates and accountability (Scale 1-5)} & Public official accountability &\cellcolor{green!15}4.8 \\\hline
                                   & Targeting &\cellcolor{green!15}4.8 \\\cdashline{2-3}
                                   & Monitoring &\cellcolor{green!15}4.3 \\\cdashline{2-3}
                                   & Incentives &\cellcolor{yellow!15}3.8 \\\cdashline{2-3}
\multirow{-4}{2cm}{National learning goals (Scale 1-5)} & Community engagement &\cellcolor{green!15}4 \\\hline
                                   & Per child spending adequacy &0 \\\cdashline{2-3}
                                   & Public expenditure and financial accountability efficiency &0.6 \\\cdashline{2-3}
                                   & Financing and efficiency outcome &0.8 \\\cdashline{2-3}
\multirow{-4}{2cm}{Financing (Scale 0-1)} & Equity &NA \\\hline
\end{tabular}}
\\
\setstretch{0.8}\color{darkgray}\scriptsize{\textit{Notes:} Financing sub-indicators are on a 0-1 scale, other sub-indicators on 0-5 scale.}
\end{table}
\raggedbottom

{\scriptsize
    \textcolor{darkgray}{\textit{Disclaimer:} GEPD numbers presented in this brief are based on multiple sources including GEPD instruments, UIS, GLAD and Learning Poverty indicators. For that reason, the numbers discussed here may be different from official statistics reported by governments and national offices of statistics. Such differences are due to the different purposes of the statistics, which can be for global comparison or to meet national definitions.}
  }

\end{document}
