\PassOptionsToPackage{unicode=true}{hyperref} % options for packages loaded elsewhere
\PassOptionsToPackage{hyphens}{url}
%
\documentclass[twocolumn]{article}
\usepackage{lmodern}
\usepackage{amssymb,amsmath}
\usepackage{ifxetex,ifluatex}
\usepackage{fixltx2e} % provides \textsubscript
\ifnum 0\ifxetex 1\fi\ifluatex 1\fi=0 % if pdftex
  \usepackage[T1]{fontenc}
  \usepackage[utf8]{inputenc}
  \usepackage{textcomp} % provides euro and other symbols
\else % if luatex or xelatex
  \usepackage{unicode-math}
  \defaultfontfeatures{Ligatures=TeX,Scale=MatchLowercase}
\fi
% use upquote if available, for straight quotes in verbatim environments
\IfFileExists{upquote.sty}{\usepackage{upquote}}{}
% use microtype if available
\IfFileExists{microtype.sty}{%
\usepackage[]{microtype}
\UseMicrotypeSet[protrusion]{basicmath} % disable protrusion for tt fonts
}{}
\IfFileExists{parskip.sty}{%
\usepackage{parskip}
}{% else
\setlength{\parindent}{0pt}
\setlength{\parskip}{6pt plus 2pt minus 1pt}
}
\usepackage{hyperref}
\hypersetup{
            pdfborder={0 0 0},
            breaklinks=true}
\urlstyle{same}  % don't use monospace font for urls
\usepackage[left=0.5cm,right=0.5cm,top=1.5cm,bottom=1.2cm]{geometry}
\usepackage{graphicx,grffile}
\makeatletter
\def\maxwidth{\ifdim\Gin@nat@width>\linewidth\linewidth\else\Gin@nat@width\fi}
\def\maxheight{\ifdim\Gin@nat@height>\textheight\textheight\else\Gin@nat@height\fi}
\makeatother
% Scale images if necessary, so that they will not overflow the page
% margins by default, and it is still possible to overwrite the defaults
% using explicit options in \includegraphics[width, height, ...]{}
\setkeys{Gin}{width=\maxwidth,height=\maxheight,keepaspectratio}
\setlength{\emergencystretch}{3em}  % prevent overfull lines
\providecommand{\tightlist}{%
  \setlength{\itemsep}{0pt}\setlength{\parskip}{0pt}}
\setcounter{secnumdepth}{0}
% Redefines (sub)paragraphs to behave more like sections
\ifx\paragraph\undefined\else
\let\oldparagraph\paragraph
\renewcommand{\paragraph}[1]{\oldparagraph{#1}\mbox{}}
\fi
\ifx\subparagraph\undefined\else
\let\oldsubparagraph\subparagraph
\renewcommand{\subparagraph}[1]{\oldsubparagraph{#1}\mbox{}}
\fi

% set default figure placement to htbp
\makeatletter
\def\fps@figure{htbp}
\makeatother

\usepackage{fancyhdr} \usepackage{booktabs,xcolor} \pagestyle{fancy} \renewcommand{\headrulewidth}{0pt} \rhead{2019} \fancypagestyle{plain}{\pagestyle{fancy}} \setlength{\headheight}{77.3pt} \setlength{\footskip}{35.1pt} \setlength{\textheight}{0.90\textheight} \pagenumbering{gobble} \usepackage[fontsize=9pt]{scrextend} \usepackage{float} \restylefloat{table} \usepackage{xcolor} \usepackage{multicol} \usepackage{array} \usepackage{colortbl} \usepackage{multirow} \usepackage{collcell} \usepackage{setspace} \usepackage{arydshln} \usepackage{caption} \captionsetup{skip=0pt} \setlength{\columnsep}{1.5cm} \setstretch{1}

\author{}
\date{\vspace{-2.5em}}

\begin{document}

\newcommand{\greynote}[1]{
    {\scriptsize
    \textcolor{darkgray}{\textit{Source:} #1}
  }
}

\newcommand{\greysource}[1]{
    {\scriptsize
    \textcolor{darkgray}{\textit{Source:} #1}
  }
}

\newcommand{\greydisclaimer}[1]{
    {\scriptsize
    \textcolor{darkgray}{\textit{Disclaimer:} #1}
  }
}

\newcommand{\greytext}[1]{
    {\scriptsize
    \textcolor{darkgray}{#1}
  }
}

\newcommand*{\tabindent}{\hspace{1mm}}

\hypertarget{introduction}{%
\subsubsection{\texorpdfstring{\textbf{INTRODUCTION}}{INTRODUCTION}}\label{introduction}}

\textbf{The Global Education Policy Dashboard (GEPD): An innovative tool
to measure drivers of learning outcomes in basic education}\\
GEPD uses 3 data collection instruments to report on nearly 40
indicators that operationalize the World Development Report 2018
framework to track 3 areas for progress in education- Practices,
Policies, and Politics. Using these indicators, the dashboard highlights
areas where countries need to act to improve learning outcomes and
allows a way for governments to track progress as they act to close gaps
in these areas. For more information on GEPD, please visit
\textbf{\href{https://www.worldbank.org/en/topic/education/brief/global-education-policy-dashboard}{www.worldbank.org/global-education-policy-dashboard}}

\hypertarget{figure-1.-gepd-framework-practices-policies-and-politics-expanding-on-wdr-2018-framework}{%
\paragraph{Figure 1. GEPD framework (practices, policies and politics),
expanding on WDR 2018
framework}\label{figure-1.-gepd-framework-practices-policies-and-politics-expanding-on-wdr-2018-framework}}

\begin{center}\includegraphics[width=0.8\linewidth]{C:/Users/wb469649/OneDrive - WBG/Documents/Github/GEPD/Country_Reports/Data/full_circle} \end{center}

\hypertarget{instruments-of-data}{%
\subsubsection{\texorpdfstring{\textbf{INSTRUMENTS OF
DATA}}{INSTRUMENTS OF DATA}}\label{instruments-of-data}}

The \textbf{School Survey} consists of 8 modules to collect data across
200-300 schools on practices (the quality of service delivery in
schools) and de facto policy indicators. It consists of streamlined
versions of existing instruments together with new questions to fill
gaps in those instruments.\\
The \textbf{Policy Survey} collects information via interviews with
\textasciitilde{}200 officials per country at federal and regional level
to feed into the policy de jure indicators and identify key elements of
the policy framework.\\
The \textbf{Survey of public officials} collects information about the
capacity and orientation of the bureaucracy and political factors
affecting education outcomes. This survey is an education-focused
version of the civil-servant surveys from the Bureaucracy Lab, WBG.

\hypertarget{key-takeaways-peru-2019}{%
\subsubsection{\texorpdfstring{\textbf{KEY TAKEAWAYS, PERU,
2019}}{KEY TAKEAWAYS, PERU, 2019}}\label{key-takeaways-peru-2019}}

\begin{itemize}
\tightlist
\item
  52\% learning poverty is observed at grade 4 level
\item
  Overall Grade 4 student proficiency (\textgreater{}80\% student
  knowledge) is only 33\%, attributed to low numeracy proficiency 23\%).
\item
  Low teacher content knowledge attributed to poor teacher support and
  lack of strong instructional leadership (especially principal feedback
  on classroom observation)
\item
  Low capacity for learning in Grade 1 (particularly socio-emotional
  knowledge) is attributed to low enrolment in early childhood programs
  (13\%)
\item
  A \textasciitilde{}0.8 point difference exists in the policy de-jure
  and de-facto indicators on a 5 point scale for teaching and school
  management. Major gaps are seen in teacher support, teacher motivation
  and evaluation of school management
\item
  Bureaucracy scores high on mandates and accountability but low on
  impartial decision making
\end{itemize}

\setlength\dashlinedash{0.2pt}
\setlength\dashlinegap{1.5pt}
\setlength\arrayrulewidth{0.3pt}
\definecolor{outcome}{RGB}{231, 128, 65}
\definecolor{practice}{RGB}{69, 164, 211}
\definecolor{policy}{RGB}{37, 117, 186}
\definecolor{politics}{RGB}{0, 35, 69}

\hypertarget{table-1.-gepd-indicators-peru-2019} \\\cdashline{3-4}
\cellcolor{outcome} &                             & Proficiency by End of Primary  & {\%} \\\cdashline{3-4}
\cellcolor{outcome} &                             & Proficiency on GEPD Assessment & {\cellcolor{red!30}33\%} \\\cdashline{2-4}
\cellcolor{outcome} &  Participation              & Net Adjusted Enrollment Rate   & {\%} \\\cdashline{2-4}
\cellcolor{outcome}\multirow{-5}{*}{\rotatebox{90}{\textcolor{white}{Outcomes}}} &  Participation and learning  & Learning Poverty Rate & {\cellcolor{red!30}56\%} \\\cdashline{1-4}
\cellcolor{practice} & \multirow{3}{2cm}{Teaching}                 & Teacher Effort           & {\cellcolor{green!30}91\%} \\\cdashline{3-4}
\cellcolor{practice} &                                             & Content Knowledge        & {\cellcolor{red!30}39\%} \\\cdashline{3-4}
\cellcolor{practice} &                                             & Pedagogical Skills       & {\cellcolor{red!30}34\%} \\\cdashline{2-4}
\cellcolor{practice} & \multirow{2}{2cm}{Inputs \& Infrastructure} & Basic Inputs             & \cellcolor{green!30}4.1 \\\cdashline{3-4}
\cellcolor{practice} &                                             & Basic Infrastructure     & \cellcolor{yellow!30}3.7 \\\cdashline{2-4}
\cellcolor{practice} & \multirow{2}{2cm}{Learners}                 & Capacity for Learning    & {\cellcolor{red!30}53\%} \\\cdashline{3-4}
\cellcolor{practice} &                                             & Student Attendance       & {\cellcolor{green!30}93\%} \\\cdashline{2-4}
\cellcolor{practice} & \multirow{4}{2cm}{School Management}        & Operational Management   & \cellcolor{green!30}4.3 \\\cdashline{3-4}
\cellcolor{practice} &                                             & Instructional Leadership & \cellcolor{yellow!30}3.3 \\\cdashline{3-4}
\cellcolor{practice} &                                             & School Knowledge         & \cellcolor{yellow!30}3.6 \\\cdashline{3-4}
\cellcolor{practice}\multirow{-13}{*}{\rotatebox{90}{\textcolor{white}{Practices}}}  & & Management Practices  & \cellcolor{green!30}4.3 \\\cdashline{1-4}
\cellcolor{policy}   & \multirow{6}{2cm}{Teaching}                  & Attraction                    & \cellcolor{yellow!30}3.6 \\\cdashline{3-4}
\cellcolor{policy}   &                                              & Selection \& Deployment       & \cellcolor{yellow!30}3.6 \\\cdashline{3-4}
\cellcolor{policy}   &                                              & Support                       & \cellcolor{yellow!30}3.1 \\\cdashline{3-4}
\cellcolor{policy}   &                                              & Evaluation                    & \cellcolor{green!30}4 \\\cdashline{3-4}
\cellcolor{policy}   &                                              & Monitoring \& Accountability  & \cellcolor{red!30}2.8 \\\cdashline{3-4}
\cellcolor{policy}   &                                              & Intrinsic Motivation          & \cellcolor{green!30}4.1 \\\cdashline{2-4}
\cellcolor{policy}   & \multirow{2}{2cm}{Inputs \& infrastructure}  & Standards                     & \cellcolor{yellow!30}3.7 \\\cdashline{3-4} 
\cellcolor{policy}   &                                              & Monitoring                    & \cellcolor{yellow!30}3.7 \\\cdashline{2-4}
\cellcolor{policy}   & \multirow{5}{2cm}{Learners}                  & Nutrition Programs            & \cellcolor{yellow!30}3.6 \\\cdashline{3-4}
\cellcolor{policy}   &                                              & Health Programs               & \cellcolor{yellow!30}3.6 \\\cdashline{3-4}
\cellcolor{policy}   &                                              & Center based care             & \cellcolor{red!30}2.6 \\\cdashline{3-4}
\cellcolor{policy}   &                                              & Caregiver Financial Capacity  & \cellcolor{red!30}2.6 \\\cdashline{3-4}
\cellcolor{policy}   &                                              & Caregiver Skills Capacity     & \cellcolor{red!30}2.9 \\\cdashline{2-4}
\cellcolor{policy}   & \multirow{5}{2cm}{School Management}         & Clarity of Functions          & \cellcolor{green!30}5 \\\cdashline{3-4}
\cellcolor{policy}   &                                              & Attraction                    & \cellcolor{green!30}4.4 \\\cdashline{3-4}
\cellcolor{policy}   &                                              & Selection \& Deployment       & \cellcolor{green!30}4.4 \\\cdashline{3-4}
\cellcolor{policy}   &                                              & Support                       & \cellcolor{yellow!30}3.5 \\\cdashline{3-4}
\cellcolor{policy}\multirow{-18}{*}{\rotatebox{90}{\textcolor{white}{Policy levers}}} & & Evaluation           & \cellcolor{green!30}4.4 \\\cdashline{1-4}
\cellcolor{politics} & \multirow{5}{2cm}{Politics \& Bureaucratic Capacity}           & Quality of Bureaucracy & \cellcolor{yellow!30}3.6 \\\cdashline{3-4}
\cellcolor{politics} & & Impartial Decision-Making  & \cellcolor{yellow!30}3.6 \\\cdashline{3-4}
\cellcolor{politics} & & Mandates \& Accountability & \cellcolor{yellow!30}3.1 \\\cdashline{3-4}
\cellcolor{politics} & & National Learning Goals    & \cellcolor{green!30}4 \\\cdashline{3-4}
\cellcolor{politics}\multirow{-5}{*}{\rotatebox{90}{\textcolor{white}{Politics}}}     & & Financing            & \cellcolor{red!30}2.8 \\\hline
\end{tabular}}
\\
{\scriptsize
    \textcolor{darkgray}{\textit{Source:} UIS, GLAD, GEPD, World Bank, Peru, 2019. For information on indicators, please consult the World Bank \href{https://github.com/worldbank/GEPD}{\underline{GEPD}}, \href{https://github.com/worldbank/GLAD}{\underline{GLAD}} and \href{https://github.com/worldbank/LearningPoverty}{\underline{Learning Poverty}} repositories.}
  }
\\
{\scriptsize
    \textcolor{darkgray}{\textit{Source:} All indicators are on a scale of 0-5 unless measured in \%. Green indicates indicator 'on-target', yellow indicates 'requires caution', red indicates 'needs improvement'}
  }

\end{table}
\vfill\null

\hypertarget{table-2.-gepd-indicators-by-gender-peru-2019} & {\cellcolor{red!30}31\%} \\\cdashline{1-4}
\multirow{3}{*}{Teaching} & Teacher Effort      & {\cellcolor{green!30}89\%} & {\cellcolor{green!30}91\%} \\\cdashline{2-4}
                          &  Content knowledge  & {\cellcolor{red!30}38\%} & {\cellcolor{red!30}39\%} \\\cdashline{2-4}
                          &  Pedagogical Skills & {NA\%} & {NA\%} \\\cdashline{1-4}
\multirow{2}{*}{Learners}   & Capacity for Learning    & {\cellcolor{yellow!30}76\%} & {\cellcolor{yellow!30}78\%} \\\cdashline{2-4}
                            & Student Attendance       & {\cellcolor{green!30}93\%} & {\cellcolor{green!30}91\%} \\\cdashline{1-4}
\multirow{4}{2cm}{School Management}  & Operational Management   & \cellcolor{green!30}4.3 & \cellcolor{green!30}4.3 \\\cdashline{2-4}
                                      & Instructional Leadership & \cellcolor{yellow!30}3.3 & \cellcolor{yellow!30}3.2 \\\cdashline{2-4}
                                      & School Knowledge         & \cellcolor{yellow!30}3.5 & \cellcolor{yellow!30}3.8 \\\cdashline{2-4}
                                      & Management Practices     & \cellcolor{green!30}4.3 & \cellcolor{green!30}4.2\\\hline
\end{tabular}}
\end{table}

\hypertarget{table-3.-gepd-indicators-by-region-peru-2019} & {\cellcolor{red!30}37\%} \\\cdashline{1-4}
\multirow{3}{*}{Teaching} & Teacher Effort                 & {\cellcolor{green!30}91\%} & {\cellcolor{green!30}91\%} \\\cdashline{2-4}
                          &  Content knowledge             & {\cellcolor{red!30}29\%} & {\cellcolor{red!30}41\%} \\\cdashline{2-4}
                          &  Pedagogical Skills            & {\cellcolor{red!30}22\%} & {\cellcolor{red!30}36\%} \\\cdashline{1-4}
\multirow{2}{2cm}{Inputs \& infrastructure}   & Basic Inputs             & \cellcolor{yellow!30}3.8 & \cellcolor{green!30}4.1 \\\cdashline{2-4}
                                              & Basic Infrastructure     & \cellcolor{red!30}2.7 & \cellcolor{green!30}4 \\\cdashline{1-4}
\multirow{2}{*}{Learners}   & Capacity for Learning    & {\cellcolor{yellow!30}63\%} & {\cellcolor{green!30}80\%} \\\cdashline{2-4}
                            & Student Attendance       & {\cellcolor{green!30}92\%} & {\cellcolor{green!30}93\%} \\\cdashline{1-4}
\multirow{4}{2cm}{School Management}  & Operational Management   & \cellcolor{green!30}4.3 & \cellcolor{green!30}4.3 \\\cdashline{2-4}
                                      & Instructional Leadership & \cellcolor{yellow!30}3.1 & \cellcolor{yellow!30}3.3 \\\cdashline{2-4}
                                      & School Knowledge         & \cellcolor{yellow!30}3.8 & \cellcolor{yellow!30}3.6 \\\cdashline{2-4}
                                      & Management Practices     & \cellcolor{green!30}4.2 & \cellcolor{green!30}4.3\\\hline
\end{tabular}}
\end{table}

\hypertarget{overall-learning-outcomes-56-learning-poverty-in-grade-4}{%
\subsubsection{\texorpdfstring{\textbf{OVERALL LEARNING OUTCOMES: 56\%
LEARNING POVERTY IN GRADE
4}}{OVERALL LEARNING OUTCOMES: 56\% LEARNING POVERTY IN GRADE 4}}\label{overall-learning-outcomes-56-learning-poverty-in-grade-4}}

Learning poverty is 5\% points worse, NA than the average for Middle
East \& North Africa (excluding high income), NA region and 27\% points
worse, NA than the average for Lower middle income, NA countries.

\hypertarget{figure-2.-leaning-poverty-peru}{%
\paragraph{Figure 2. Leaning poverty,
Peru}\label{figure-2.-leaning-poverty-peru}}

\includegraphics{C:/Users/wb469649/OneDrive - WBG/Documents/Github/GEPD/Country_Reports/Output/Peru_2019_4Pager_files/figure-latex/lp_fig-1.pdf}

Average proficiency in Grade 4 is 38 points higher in literacy compared
to numeracy,2 points higher for boys compared to girls, and 23 points
higher in urban areas compared to rural areas.

\hypertarget{figure-3.-grade-4-proficiency-peru}{%
\paragraph{Figure 3. Grade 4 proficiency,
Peru}\label{figure-3.-grade-4-proficiency-peru}}

\includegraphics{C:/Users/wb469649/OneDrive - WBG/Documents/Github/GEPD/Country_Reports/Output/Peru_2019_4Pager_files/figure-latex/lern_teach-1.pdf}

\textbf{Teacher effectiveness}\\
Teacher content knowledge (39\%) needs improvement. Teacher proficiency
in language (41\%) is 5 points higher than mathematics proficiency
(36\%). Teacher pedagogical skills score (34\%) needs improvement.

\begin{table}[H]
\resizebox{\columnwidth}{!}{\begin{tabular}{m{3.8cm}cm{4.2cm}c}
\multicolumn{2}{c}{\textbf{Practice Indicators}} & \multicolumn{2}{c}{\textbf{Policy lever indicators}}\\\hline
Content knowledge                   & {\cellcolor{red!30}39\%}   & Attraction                                & \cellcolor{yellow!30}3.6 \\\cdashline{1-4} 
\hspace{1mm}\emph{Maths proficiency}               & {\cellcolor{red!30}36\%} &                                           & \cellcolor{yellow!30}\\\cdashline{1-2}   
\hspace{1mm}\emph{Language proficiency}            & {\cellcolor{red!30}41\%} & \multirow{-2}{*}{Selection \& deployment} & \multirow{-2}{*}{\cellcolor{yellow!30}3.6} \\\cdashline{1-4}        
Pedagogical skills                               & {\cellcolor{red!30}34\%}   &                                           & \cellcolor{yellow!30}\\\cdashline{1-2}
\hspace{1mm}\emph{\% Classroom culture}            & {\cellcolor{green!30}91\%} & \multirow{-2}{*}{Support}                 & \multirow{-2}{*}{\cellcolor{yellow!30}3.1} \\\cdashline{1-4}
\hspace{1mm}\emph{\% Instruction practices}        & {\cellcolor{red!30}39\%} &                                           & \cellcolor{green!30}\\\cdashline{1-2}
\hspace{1mm}\emph{\% Socio-emotional skills}       & {\cellcolor{red!30}24\%} & \multirow{-2}{*}{Evaluation}              & \multirow{-2}{*}{\cellcolor{green!30}4} \\\cdashline{1-4}
& \cellcolor{green!30} & Monitoring \& Accountability  & \cellcolor{red!30}2.8 \\\cdashline{3-4}
\multirow{-2}{*}{Teacher Attendance}             & \multirow{-2}{*}{\cellcolor{green!30}92\%} & Intrinsic motivation    & \cellcolor{green!30}4.1 \\\hline
\end{tabular}}
\\
{\scriptsize
    \textcolor{darkgray}{\textit{Source:} Content knowledge(\& sub-indicators) indicate \% teachers with knowledge\textgreater{80\%}.Pedagogical skills(\& sub-indicators) indicate \% teachers with proficiency 3/5 or above.}
  }

\end{table}
\raggedbottom

\textbf{Capacity for learning}\\
Student proficiency in Grade 1 (53\%) needs improvement. NA is the
lowest knoweldge sub-score (65\%). Student attendance(93\%) is on
target.

\begin{table}[H]
\resizebox{\columnwidth}{!}{\begin{tabular}{m{3.8cm}cm{4.2cm}c}
\multicolumn{2}{c}{\textbf{Practice Indicators}} & \multicolumn{2}{c}{\textbf{Policy lever indicators}}\\\hline
Capacity for learning                  & {\cellcolor{red!30}53\%} & & \cellcolor{green!30}\\\cdashline{1-2}
\hspace{1mm}\emph{Numeracy score}        & \cellcolor{green!30}90   & \multirow{-2}{*}{Nutrition Programs} & \multirow{-2}{*}{\cellcolor{green!30}4.3}  \\\cdashline{1-4}  
\hspace{1mm}\emph{Literacy score}        & \cellcolor{yellow!30}77   & Health Programs                     & \cellcolor{green!30}4.2 \\\cdashline{1-4}  
\hspace{1mm}\emph{Executive score}       & \cellcolor{yellow!30}65   & Center based care                   & \cellcolor{red!30}2.6 \\\cdashline{1-4}  
\hspace{1mm}\emph{Socio-emotional score} & \cellcolor{yellow!30}75   & Caregiver Skills Capacity           & \cellcolor{red!30}2.9 \\\cdashline{1-4}  
Student Attendance                     & {\cellcolor{green!30}93\%}     & Caregiver Financial Capacity        & \cellcolor{green!30}4.5 \\\hline
\end{tabular}}
\\
{\scriptsize
    \textcolor{darkgray}{\textit{Source:} Capacity for learning indicates \% students with knowledge\textgreater{80\%}. Sub-indicator scores refer to average subject knowledge on a 0-100 scale.}
  }

\end{table}
\raggedbottom

\textbf{Inputs and infrastructure}\\
Basic inputs (4/5) are on target. NA is the lowest score (78\%). Basic
infrastructure (4/5) is on target. NA is the lowest score (54\%).

\begin{table}[H]
\resizebox{\columnwidth}{!}{\begin{tabular}{m{3.8cm}cm{4.2cm}c}
\multicolumn{2}{c}{\textbf{Practice Indicators}} & \multicolumn{2}{c}{\textbf{Policy lever indicators}}\\\hline
Basic inputs & \cellcolor{green!30}4.1 & & \cellcolor{yellow!30} \\\cdashline{1-2}
\hspace{1mm}\emph{\% Blackboard}    & {\cellcolor{green!30}82\%} & & \cellcolor{yellow!30}\\\cdashline{1-2}
\hspace{1mm}\emph{\% Stationery}    & {\cellcolor{green!30}96\%} & & \cellcolor{yellow!30}\\\cdashline{1-2}
\hspace{1mm}\emph{\% Furniture}     & {\cellcolor{green!30}84\%} & & \cellcolor{yellow!30}\\\cdashline{1-2}
\hspace{1mm}\emph{\% EdTech access} & {\cellcolor{yellow!30}78\%} & \multirow{-5}{4.2cm}{Inputs and infrastructure standards} & \multirow{-5}{*}{\cellcolor{yellow!30}3.7}\\\hline
Basic infrastructure & \cellcolor{yellow!30}3.7 & & \cellcolor{yellow!30} \\\cdashline{1-2}
\hspace{1mm}\emph{\% Drinking water}    & {\cellcolor{green!30}91\%} & & \cellcolor{yellow!30}\\\cdashline{1-2}
\hspace{1mm}\emph{\% Functional toilet} & {\cellcolor{yellow!30}68\%} & & \cellcolor{yellow!30}\\\cdashline{1-2}
\hspace{1mm}\emph{\% Internet}          & {\cellcolor{green!30}96\%} & & \cellcolor{yellow!30}\\\cdashline{1-2}
\hspace{1mm}\emph{\% Electricity}       & {\cellcolor{red!30}54\%} & & \cellcolor{yellow!30}\\\cdashline{1-2}
\hspace{1mm}\emph{\% Disability access} & {\cellcolor{yellow!30}65\%} & \multirow{-6}{4.2cm}{Inputs and infrastructure monitoring} & \multirow{-6}{*}{\cellcolor{yellow!30}3.7}\\\hline
\end{tabular}}
\\
{\scriptsize
    \textcolor{darkgray}{\textit{Source:} \% refers to \% schools with the given sub-component}
  }

\end{table}
\raggedbottom

\textbf{School Management by principals}\\
In school management, the lowest score is for principal's Instructional
Leadership(3.3/5), whereas the highest score is obtained for Operational
Management(4.3/5)

\begin{table}[H]
\resizebox{\columnwidth}{!}{\begin{tabular}{m{4.8cm}cm{4cm}c}
\multicolumn{2}{c}{\textbf{Practice Indicators}} & \multicolumn{2}{c}{\textbf{Policy lever indicators}}\\\hline
Operational management                     & \cellcolor{green!30}4.3       & & \cellcolor{green!30}\\\cdashline{1-2}
\hspace{1mm}\emph{Infrastructure}            & \cellcolor{green!30}4.2     & & \cellcolor{green!30}\\\cdashline{1-2}
\hspace{1mm}\emph{Ensuring inputs}           & \cellcolor{green!30}4.4     & & \cellcolor{green!30}\\\cdashline{1-2}
Instructional Leadership                   & \cellcolor{yellow!30}3.3       & & \cellcolor{green!30}\\\cdashline{1-2}
\hspace{1mm}\emph{\% Classroom observed}     & {\cellcolor{green!30}86\%} & & \cellcolor{green!30}\\\cdashline{1-2}
\hspace{1mm}\emph{\% Discussed observations} & {\cellcolor{red!30}24\%} & & \cellcolor{green!30}\\\cdashline{1-2}
\hspace{1mm}\emph{\% Feedback given}         & {\cellcolor{yellow!30}71\%} & & \cellcolor{green!30}\\\cdashline{1-2}
\hspace{1mm}\emph{\% Lesson-plan feedback}   & {\cellcolor{yellow!30}73\%} & \multirow{-8}{*}{Clarity of functions} & \multirow{-8}{*}{\cellcolor{green!30}5} \\\hline
Principal school knowledge                         & \cellcolor{yellow!30}3.6       & & \cellcolor{green!30}\\\cdashline{1-2}
\hspace{1mm}\emph{\% Familiarity-teacher knowledge}  & {\cellcolor{yellow!30}75\%} & \multirow{-2}{4cm}{Attraction} & \multirow{-2}{*}{\cellcolor{green!30}4.4}\\\cdashline{1-4}
\hspace{1mm}\emph{\% Familiarity-teacher experience} & {\cellcolor{green!30}99\%} & & \cellcolor{green!30}\\\cdashline{1-2}
\hspace{1mm}\emph{\% Familiarity-school inputs}      & {\cellcolor{green!30}81\%} & \multirow{-2}{4cm}{Selection \& Deployment} & \multirow{-2}{*}{\cellcolor{green!30}4.4} \\\cdashline{1-4} 
Management skills & \cellcolor{green!30}4.3 & & \cellcolor{yellow!30}3.5\\\cdashline{1-2}
\hspace{1mm}\emph{Problem solving} & \cellcolor{green!30}4.6 & \multirow{-2}{4cm}{Support} & \multirow{-2}{*}{\cellcolor{yellow!30}3.5}\\\cdashline{1-4}
\hspace{1mm}\emph{Goal-setting} & \cellcolor{yellow!30}3.9 & Evaluation & \cellcolor{green!30}4.4\\\hline
\end{tabular}}
\\
{\scriptsize
    \textcolor{darkgray}{\textit{Source:} \% refers to \% schools with the given sub-component}
  }

\end{table}
\raggedbottom

\hypertarget{gaps-between-de-facto-indicators-and-de-jure-poicies-in-teaching-and-school-management}{%
\subsubsection{\texorpdfstring{\textbf{GAPS BETWEEN DE-FACTO INDICATORS
AND DE-JURE POICIES IN TEACHING AND SCHOOL
MANAGEMENT}}{GAPS BETWEEN DE-FACTO INDICATORS AND DE-JURE POICIES IN TEACHING AND SCHOOL MANAGEMENT}}\label{gaps-between-de-facto-indicators-and-de-jure-poicies-in-teaching-and-school-management}}

A \textasciitilde{}0.3 point average gap exists between the de-facto
indicators and de-jure policies in Peru across teaching and school
management.

\hypertarget{figure-4.-de-facto-and-de-jure-indicators-for-policy-levers-peru}{%
\paragraph{Figure 4. De-facto and de-jure indicators for policy levers,
Peru}\label{figure-4.-de-facto-and-de-jure-indicators-for-policy-levers-peru}}

\includegraphics{C:/Users/wb469649/OneDrive - WBG/Documents/Github/GEPD/Country_Reports/Output/Peru_2019_4Pager_files/figure-latex/defacto_graphs-1.pdf}

\hypertarget{breakdown-of-sub-indicators-of-the-top-3-de-facto-indicators-which-differ-the-most-from-de-jure-indicators}{%
\subsubsection{\texorpdfstring{\textbf{Breakdown of sub-indicators of
the top 3 de-facto indicators which differ the most from de-jure
indicators}}{Breakdown of sub-indicators of the top 3 de-facto indicators which differ the most from de-jure indicators}}\label{breakdown-of-sub-indicators-of-the-top-3-de-facto-indicators-which-differ-the-most-from-de-jure-indicators}}

The highest gaps in practices and policies are observed for Teaching -
Support (1.9 points), Management- Support (1.2 points), Teaching-
Evaluation (1 points). \vfill\null

\hypertarget{figure-5-sub-indicators-of-teaching---support}{%
\paragraph{Figure 5: Sub-indicators of Teaching -
Support}\label{figure-5-sub-indicators-of-teaching---support}}

\includegraphics{C:/Users/wb469649/OneDrive - WBG/Documents/Github/GEPD/Country_Reports/Output/Peru_2019_4Pager_files/figure-latex/defacto_sub_graphs_1-1.pdf}

\setstretch{0.8}\color{darkgray}\scriptsize{\textit{Notes:} Percent of teachers reporting in affirmative unless stated otherwise}

\hypertarget{figure-6-sub-indicators-of-management--support}{%
\paragraph{Figure 6: Sub-indicators of Management-
Support}\label{figure-6-sub-indicators-of-management--support}}

\includegraphics{C:/Users/wb469649/OneDrive - WBG/Documents/Github/GEPD/Country_Reports/Output/Peru_2019_4Pager_files/figure-latex/defacto_sub_graphs_2-1.pdf}

\setstretch{0.8}\color{darkgray}\scriptsize{\textit{Notes:} Percent of principals reporting on the given school management support aspect.}

\hypertarget{figure-7-sub-indicators-of-teaching--evaluation}{%
\paragraph{Figure 7: Sub-indicators of Teaching-
Evaluation}\label{figure-7-sub-indicators-of-teaching--evaluation}}

\includegraphics{C:/Users/wb469649/OneDrive - WBG/Documents/Github/GEPD/Country_Reports/Output/Peru_2019_4Pager_files/figure-latex/defacto_sub_graphs_3-1.pdf}

\setstretch{0.8}\color{darkgray}\scriptsize{\textit{Notes:} Percent of teachers reporting on the given teacher evaluation aspect.}

\vfill\null

\hypertarget{politics-bureaucratic-capacity-indicators}{%
\subsubsection{\texorpdfstring{\textbf{POLITICS \& BUREAUCRATIC CAPACITY
INDICATORS}}{POLITICS \& BUREAUCRATIC CAPACITY INDICATORS}}\label{politics-bureaucratic-capacity-indicators}}

\singlespacing{The highest score in politics and bureaucratic capacity is noted for  (5/5), and the lowest score is noted for  (3/5).}

\hypertarget{figure-8.-politics-and-bureaucratic-capacity-indicators-peru}{%
\paragraph{Figure 8. Politics and bureaucratic capacity indicators,
Peru}\label{figure-8.-politics-and-bureaucratic-capacity-indicators-peru}}

\includegraphics{C:/Users/wb469649/OneDrive - WBG/Documents/Github/GEPD/Country_Reports/Output/Peru_2019_4Pager_files/figure-latex/politics_graph-1.pdf}

\hypertarget{table-4.-politics-and-bureaucratic-capacity-sub-indicators-peru}{%
\subsubsection{Table 4. Politics and bureaucratic capacity
sub-indicators,
Peru}\label{table-4.-politics-and-bureaucratic-capacity-sub-indicators-peru}}

\begin{table}[H]
\resizebox{\columnwidth}{!}{\begin{tabular}{p{7cm}c}
\textbf{Indicator} & \textbf{Value} \\ \hline
\textbf\textit{Quality of bureaucracy (Scale 1-5)} & \\
Knowledge and skills &\cellcolor{yellow!30}3.9 \\\cdashline{1-2}
Work enviornment &\cellcolor{green!30}4 \\\cdashline{1-2}
Merit &\cellcolor{yellow!30}3.2 \\\cdashline{1-2}
Motivation and attitudes &\cellcolor{green!30}4.3 \\\hline
\textbf\textit{Impartial decision making (Scale 1-5)} & \\
Politicized personnel management &\cellcolor{yellow!30}3.5 \\\cdashline{1-2}
Politicized policy-making &\cellcolor{yellow!30}3.6 \\\cdashline{1-2}
Politicized policy implementation &\cellcolor{red!30}2.9 \\\cdashline{1-2}
Employee unions as facilitators &\cellcolor{green!30}4 \\\hline
\textbf\textit{Mandates and accountability (Scale 1-5)} & \\
Coherence &\cellcolor{green!30}4 \\\cdashline{1-2}
Transparency &\cellcolor{green!30}4.1 \\\cdashline{1-2}
Public official accountability &\cellcolor{green!30}4.6 \\\hline
\textbf\textit{National learning goals (Scale 1-5)} & \\
Targeting &\cellcolor{green!30}4.2 \\\cdashline{1-2}
Monitoring &\cellcolor{yellow!30}3.7 \\\cdashline{1-2}
Incentives &\cellcolor{red!30}2.6 \\\cdashline{1-2}
Community engagement &\cellcolor{red!30}2.4 \\\hline
\textbf\textit{Financing} & \\
Per child spending adequacy(0-1) &1 \\\cdashline{1-2}
Public expenditure and financial accountability efficiency (0-1) &-999 \\\cdashline{1-2}
Financing and efficiency outcome (0-1) &-999 \\\cdashline{1-2}
Equity &-999 \\\hline
\end{tabular}}
\end{table}

\end{document}
