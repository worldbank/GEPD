\PassOptionsToPackage{unicode=true}{hyperref} % options for packages loaded elsewhere
\PassOptionsToPackage{hyphens}{url}
%
\documentclass[twocolumn]{article}
\usepackage{lmodern}
\usepackage{amssymb,amsmath}
\usepackage{ifxetex,ifluatex}
\usepackage{fixltx2e} % provides \textsubscript
\ifnum 0\ifxetex 1\fi\ifluatex 1\fi=0 % if pdftex
  \usepackage[T1]{fontenc}
  \usepackage[utf8]{inputenc}
  \usepackage{textcomp} % provides euro and other symbols
\else % if luatex or xelatex
  \usepackage{unicode-math}
  \defaultfontfeatures{Ligatures=TeX,Scale=MatchLowercase}
\fi
% use upquote if available, for straight quotes in verbatim environments
\IfFileExists{upquote.sty}{\usepackage{upquote}}{}
% use microtype if available
\IfFileExists{microtype.sty}{%
\usepackage[]{microtype}
\UseMicrotypeSet[protrusion]{basicmath} % disable protrusion for tt fonts
}{}
\IfFileExists{parskip.sty}{%
\usepackage{parskip}
}{% else
\setlength{\parindent}{0pt}
\setlength{\parskip}{6pt plus 2pt minus 1pt}
}
\usepackage{hyperref}
\hypersetup{
            pdfborder={0 0 0},
            breaklinks=true}
\urlstyle{same}  % don't use monospace font for urls
\usepackage[left=0.5cm,right=0.5cm,top=1.5cm,bottom=1.2cm]{geometry}
\usepackage{graphicx,grffile}
\makeatletter
\def\maxwidth{\ifdim\Gin@nat@width>\linewidth\linewidth\else\Gin@nat@width\fi}
\def\maxheight{\ifdim\Gin@nat@height>\textheight\textheight\else\Gin@nat@height\fi}
\makeatother
% Scale images if necessary, so that they will not overflow the page
% margins by default, and it is still possible to overwrite the defaults
% using explicit options in \includegraphics[width, height, ...]{}
\setkeys{Gin}{width=\maxwidth,height=\maxheight,keepaspectratio}
\setlength{\emergencystretch}{3em}  % prevent overfull lines
\providecommand{\tightlist}{%
  \setlength{\itemsep}{0pt}\setlength{\parskip}{0pt}}
\setcounter{secnumdepth}{0}
% Redefines (sub)paragraphs to behave more like sections
\ifx\paragraph\undefined\else
\let\oldparagraph\paragraph
\renewcommand{\paragraph}[1]{\oldparagraph{#1}\mbox{}}
\fi
\ifx\subparagraph\undefined\else
\let\oldsubparagraph\subparagraph
\renewcommand{\subparagraph}[1]{\oldsubparagraph{#1}\mbox{}}
\fi

% set default figure placement to htbp
\makeatletter
\def\fps@figure{htbp}
\makeatother

\usepackage{fancyhdr} \usepackage{booktabs,xcolor} \pagestyle{fancy} \renewcommand{\headrulewidth}{0pt} \rhead{2019} \cfoot{\includegraphics[width=20cm]{footer.png}} \fancypagestyle{plain}{\pagestyle{fancy}} \setlength{\headheight}{77.3pt} \setlength{\footskip}{36.5pt} \setlength{\textheight}{0.90\textheight} \pagenumbering{gobble} \usepackage[fontsize=9pt]{scrextend} \usepackage{float} \restylefloat{table} \usepackage{xcolor} \usepackage{multicol} \usepackage{array} \usepackage{colortbl} \usepackage{multirow} \usepackage{collcell} \usepackage{setspace} \usepackage{arydshln} \usepackage{caption} \captionsetup{skip=0pt} \setlength{\columnsep}{1.5cm}

\author{}
\date{\vspace{-2.5em}}

\begin{document}

\newcommand{\greynote}[1]{
    {\scriptsize
    \textcolor{darkgray}{\textit{Notes:} #1}
  }
}

\newcommand{\greysource}[1]{
    {\scriptsize
    \textcolor{darkgray}{\textit{Source:} #1}
  }
}

\newcommand{\greydisclaimer}[1]{
    {\scriptsize
    \textcolor{darkgray}{\textit{Disclaimer:} #1}
  }
}

\newcommand*{\tabindent}{\hspace{1mm}}

\hypertarget{introduction}{%
\subsubsection{\texorpdfstring{\textbf{INTRODUCTION}}{INTRODUCTION}}\label{introduction}}

\textbf{The Global Education Policy Dashboard (GEPD): An innovative tool
to measure drivers of learning outcomes in basic education}\\
GEPD uses 3 data collection instruments to report on nearly 40
indicators that operationalize the World Development Report 2018
framework to track 3 areas for progress in education- Practices,
Policies, and Politics. Using these indicators, the dashboard highlights
areas where countries need to act to improve learning outcomes and
allows a way for governments to track progress as they act to close gaps
in these areas. For more information on GEPD, please visit
\textbf{\href{https://www.worldbank.org/en/topic/education/brief/global-education-policy-dashboard}{www.worldbank.org/global-education-policy-dashboard}}

\hypertarget{figure-1.-gepd-framework-practices-policies-and-politics-expanding-on-wdr-2018-framework}{%
\paragraph{Figure 1. GEPD framework (practices, policies and politics),
expanding on WDR 2018
framework}\label{figure-1.-gepd-framework-practices-policies-and-politics-expanding-on-wdr-2018-framework}}

\begin{center}\includegraphics[width=0.8\linewidth]{C:/Users/wb469649/OneDrive - WBG/Documents/Github/GEPD/Country_Reports/Data/full_circle} \end{center}

\hypertarget{instruments-of-data}{%
\subsubsection{\texorpdfstring{\textbf{INSTRUMENTS OF
DATA}}{INSTRUMENTS OF DATA}}\label{instruments-of-data}}

The \textbf{School Survey} consists of 8 modules to collect data across
200-300 schools on practices (the quality of service delivery in
schools) and de facto policy indicators. It consists of streamlined
versions of existing instruments together with new questions to fill
gaps in those instruments.\\
The \textbf{Policy Survey} collects information via interviews with
\textasciitilde{}200 officials per country at federal and regional level
to feed into the policy de jure indicators and identify key elements of
the policy framework.\\
The \textbf{Survey of public officials} collects information about the
capacity and orientation of the bureaucracy and political factors
affecting education outcomes. This survey is an education-focused
version of the civil-servant surveys from the Bureaucracy Lab, WBG.

\hypertarget{key-takeaways-2020}{%
\subsubsection{\texorpdfstring{\textbf{KEY TAKEAWAYS,
\uppercase{Rwanda},
2020}}{KEY TAKEAWAYS, , 2020}}\label{key-takeaways-2020}}

\begin{itemize}
\tightlist
\item
  3.8 years of learning adjusted years of schooling observed in Rwanda.
  GEPD Grade 4 proficiency is low at only 0.16\%, with numeracy
  proficiency lower than language proficiency of students.
\item
  Teacher content knowledge is poor at only 27\%, attributed to poor
  teaching support and weak monitoring and accountability systems. Only
  44\% teachers reported receiving feedback from principals after
  classroom observation in schools.
\item
  Grade 1 proficiency of students is \textasciitilde{}9\%, with students
  scoring lower on executive functions and socio-emotional learning.
\item
  Basic inputs and infrastructure are weak in areas of avaiability of
  functional blackboards, functional toilets and electricity in schools.
\item
  Major gaps are seen in implementation of teaching support policies,
  teaching monitoring and accountability systems and selection and
  deployment policies for school principals.
\item
  Primary education funding amount and efficiency of spending is low and
  education policy implementation is politicized, lowering bureacucratic
  capacity.
\end{itemize}

\setlength\dashlinedash{0.2pt}
\setlength\dashlinegap{1.5pt}
\setlength\arrayrulewidth{0.3pt}

\hypertarget{table-1.-key-gepd-outcome-indicators} \\\cdashline{1-2}
Proficiency on GEPD Assessment & {\cellcolor{red!15}0.2\%} \\\cdashline{1-2}
\hspace{1mm}\emph{Literacy proficiency} & {\cellcolor{red!15}0.6\%} \\\cdashline{1-2}
\hspace{1mm}\emph{Numeracy proficiency} & {\cellcolor{red!15}0.3\%} \\\cdashline{1-2}
Proficiency by Grade 2/3 & {\cellcolor{red!15}-999\%} \\\cdashline{1-2}
Net Adjusted Enrollment Rate & {\cellcolor{green!15}95\%} \\\hline
\end{tabular}}
\\
\setstretch{0.8}\color{darkgray}\scriptsize{\textit{Source:} UIS, GLAD, GEPD, World Bank, Rwanda, 2020. For information on indicators, please consult the World Bank \href{https://github.com/worldbank/GEPD}{\underline{GEPD}}, \href{https://github.com/worldbank/GLAD}{\underline{GLAD}} and \href{https://github.com/worldbank/LearningPoverty}{\underline{Learning Poverty}} repositories.}\\
\setstretch{0.8}\color{darkgray}\scriptsize{\textit{Notes:} (1) Proficiency on GEPD assessment means \% students with knowledge\textgreater{80\%}. (2) Proficiency by end of primary uses threshold as per Minimum Proficiency Levels set by GAML(UIS). (3) All indicators are on a scale of 0-5 unless measured in \%. (4) Green indicates indicator 'on-target', yellow indicates 'requires caution', red indicates 'needs improvement'.}
\end{table}
\raggedbottom

\hypertarget{learning-outcomes-3.8-learning-adjusted-years-in-school-0.2-gepd-proficiency-in-grade-4}{%
\subsubsection{\texorpdfstring{\textbf{LEARNING OUTCOMES: 3.8 LEARNING
ADJUSTED YEARS IN SCHOOL, 0.2\% GEPD PROFICIENCY IN GRADE
4}}{LEARNING OUTCOMES: 3.8 LEARNING ADJUSTED YEARS IN SCHOOL, 0.2\% GEPD PROFICIENCY IN GRADE 4}}\label{learning-outcomes-3.8-learning-adjusted-years-in-school-0.2-gepd-proficiency-in-grade-4}}

Learning adjusted years of school (LAYS) is calculated by adjusting
expected years of schooling for schooling quality. Learning adjusted
years of schooling in Rwanda is 0.9 years lower than the average for
Sub-Saharan Africa (excluding high income) region and 0.6 years lower
than the average for Low income countries.

\hypertarget{figure-2.-learning-adjusted-years-in-school-comparison}{%
\paragraph{Figure 2. Learning adjusted years in school
comparison}\label{figure-2.-learning-adjusted-years-in-school-comparison}}

\includegraphics[width=1\linewidth]{C:/Users/wb469649/OneDrive - WBG/Documents/Github/GEPD/Country_Reports/LP figures/lays_figure_RWA}

{\scriptsize
    \textcolor{darkgray}{\textit{Notes:} Grey circles represent other countries. Yellow circle represents Rwanda. Orange and blue circles represent average LAYS in Rwanda's region and income group.}
  }

GEPD grade 4 assessment proficiency is defined as \% students scoring
greater than 80/100 on student knowledge. \textbf{GEPD grade 4
assessment proficiency of 0.2\%} means 0.2\% students score greater than
80/100 on student knowledge. Student proficiency is 1 points higher in
language compared to numeracy, 0.1 points lower for boys compared to
girls, and 0.7 points higher in urban areas compared to rural areas.

\vfill\null

\hypertarget{figure-3.-gepd-grade-4-proficiency-rwanda}{%
\paragraph{Figure 3. GEPD Grade 4 proficiency,
Rwanda}\label{figure-3.-gepd-grade-4-proficiency-rwanda}}

\includegraphics{C:/Users/wb469649/ONEDRI~1/DOCUME~1/Github/GEPD/COUNTR~1/Output/RWANDA~1/figure-latex/learning_figure-1.pdf}

\hypertarget{comparing-de-facto-practices-and-policy-levers}{%
\subsubsection{\texorpdfstring{\textbf{COMPARING DE-FACTO PRACTICES AND
POLICY
LEVERS}}{COMPARING DE-FACTO PRACTICES AND POLICY LEVERS}}\label{comparing-de-facto-practices-and-policy-levers}}

Practice indicators measure quality of service delivery in schools such
as student performance, teacher knowledge, principal management skills,
etc. Policy lever indicators measure how well school, personnel and
student policies governing these practices are implemented. Comparing
de-facto practice and policy lever indicators allows identification of
low-scoring policy levers that affect observed practice indicators.

\hypertarget{teacher-effectiveness}{%
\paragraph{\texorpdfstring{\textbf{Teacher
effectiveness}}{Teacher effectiveness}}\label{teacher-effectiveness}}

Teacher content knowledge (27\%) needs improvement. Teacher proficiency
in language (20\%) is 21 points lower than mathematics proficiency
(41\%). Teacher pedagogical skills score NA, and teacher attendance
(87\%) is on target. Teaching - Support policy lever scores the
lowest(2.8/5).

\begin{table}[H]
\resizebox{\columnwidth}{!}{\begin{tabular}{m{3.8cm}cm{4.2cm}c}
\multicolumn{2}{c}{\textbf{Practice Indicators}} & \multicolumn{2}{c}{\textbf{Policy levers (Teaching)}}\\\hline
Content knowledge                   & {\cellcolor{red!15}27\%}   & Attraction                                & \cellcolor{yellow!15}3.9 \\\cdashline{1-4} 
\hspace{1mm}\emph{Maths proficiency}               & {\cellcolor{red!15}41\%} &                                           & \cellcolor{yellow!15}\\\cdashline{1-2}   
\hspace{1mm}\emph{Language proficiency}            & {\cellcolor{red!15}20\%} & \multirow{-2}{*}{Selection \& deployment} & \multirow{-2}{*}{\cellcolor{yellow!15}3.3} \\\cdashline{1-4}        
Pedagogical skills                               & {NA\%}   &                                           & \cellcolor{red!15}\\\cdashline{1-2}
\hspace{1mm}\emph{\% Classroom culture}            & {NA\%} & \multirow{-2}{*}{Support}                 & \multirow{-2}{*}{\cellcolor{red!15}2.8} \\\cdashline{1-4}
\hspace{1mm}\emph{\% Instruction practices}        & {NA\%} &                                           & \cellcolor{green!15}\\\cdashline{1-2}
\hspace{1mm}\emph{\% Socio-emotional skills}       & {NA\%} & \multirow{-2}{*}{Evaluation}              & \multirow{-2}{*}{\cellcolor{green!15}4.5} \\\cdashline{1-4}
& \cellcolor{green!15} & Monitoring \& Accountability  & \cellcolor{red!15}2.9 \\\cdashline{3-4}
\multirow{-2}{*}{Teacher Attendance}             & \multirow{-2}{*}{\cellcolor{green!15}87\%} & Intrinsic motivation    & \cellcolor{yellow!15}3.9 \\\hline
\end{tabular}}
\\
\setstretch{0.8}\color{darkgray}\scriptsize{\textit{Notes:} Content knowledge(\& sub-indicators) indicate \% teachers with knowledge\textgreater{80\%}.Pedagogical skills(\& sub-indicators) indicate \% teachers with proficiency 3/5 or above.}
\end{table}

\hypertarget{capacity-for-learning-in-grade-1}{%
\paragraph{\texorpdfstring{\textbf{Capacity for learning in Grade
1}}{Capacity for learning in Grade 1}}\label{capacity-for-learning-in-grade-1}}

Student proficiency in Grade 1 (9\%) needs improvement. Literacy
score(38)\% is the lowest knowledge sub-score. Student attendance(87\%)
is on target. Center-Based Care policy lever scores the lowest(1.5/5).

\begin{table}[H]
\resizebox{\columnwidth}{!}{\begin{tabular}{m{3.8cm}cm{4.2cm}c}
\multicolumn{2}{c}{\textbf{Practice Indicators}} & \multicolumn{2}{c}{\textbf{Policy levers (Learners)}}\\\hline
Capacity for learning                  & {\cellcolor{red!15}9\%} & & \cellcolor{yellow!15}\\\cdashline{1-2}
\hspace{1mm}\emph{Numeracy score}        & \cellcolor{yellow!15}62   & \multirow{-2}{*}{Nutrition Programs} & \multirow{-2}{*}{\cellcolor{yellow!15}3.2}  \\\cdashline{1-4}  
\hspace{1mm}\emph{Literacy score}        & \cellcolor{red!15}38   & Health Programs                     & \cellcolor{yellow!15}3.5 \\\cdashline{1-4}  
\hspace{1mm}\emph{Executive score}       & \cellcolor{red!15}47   & Center based care                   & \cellcolor{red!15}1.5 \\\cdashline{1-4}  
\hspace{1mm}\emph{Socio-emotional score} & \cellcolor{yellow!15}67   & Caregiver Skills Capacity           & \cellcolor{red!15}2.8 \\\cdashline{1-4}  
Student Attendance                     & {\cellcolor{green!15}87\%}     & Caregiver Financial Capacity        & \cellcolor{red!15}2.3 \\\hline
\end{tabular}}
\\
\setstretch{0.8}\color{darkgray}\scriptsize{\textit{Notes:} Capacity for learning indicates \% students with knowledge\textgreater{80\%}. Sub-indicator scores refer to average subject knowledge on a 0-100 scale.}
\end{table}
\vfill\null

\hypertarget{inputs-infrastructure}{%
\paragraph{\texorpdfstring{\textbf{Inputs \&
Infrastructure}}{Inputs \& Infrastructure}}\label{inputs-infrastructure}}

Basic inputs (3.3/5) are on target. Percent of classrooms with a
functional blackboard and chalk(60)\% is the lowest score. Basic
infrastructure (3.1/5) requires caution. Percent of schools with access
to internet(27)\% is the lowest score. Inputs \& Infrastructure -
Monitoring policy lever scores the lowest(3.1/5).

\begin{table}[H]
\resizebox{\columnwidth}{!}{\begin{tabular}{m{3.8cm}cm{4.2cm}c}
\multicolumn{2}{c}{\textbf{Practice Indicators}} & \multicolumn{2}{c}{\textbf{Policy levers(Inputs)}}\\\hline
Basic inputs & \cellcolor{yellow!15}3.3 & & \cellcolor{green!15} \\\cdashline{1-2}
\hspace{1mm}\emph{\% Blackboard}    & {\cellcolor{yellow!15}60\%} & & \cellcolor{green!15}\\\cdashline{1-2}
\hspace{1mm}\emph{\% Stationery}    & {\cellcolor{green!15}92\%} & & \cellcolor{green!15}\\\cdashline{1-2}
\hspace{1mm}\emph{\% Furniture}     & {\cellcolor{green!15}99\%} & & \cellcolor{green!15}\\\cdashline{1-2}
\hspace{1mm}\emph{\% EdTech access} & {\cellcolor{yellow!15}76\%} & \multirow{-5}{4.2cm}{Inputs and infrastructure standards} & \multirow{-5}{*}{\cellcolor{green!15}4.5}\\\hline
Basic infrastructure & \cellcolor{yellow!15}3.1 & & \cellcolor{yellow!15} \\\cdashline{1-2}
\hspace{1mm}\emph{\% Drinking water}    & {\cellcolor{yellow!15}74\%} & & \cellcolor{yellow!15}\\\cdashline{1-2}
\hspace{1mm}\emph{\% Functional toilet} & {\cellcolor{red!15}57\%} & & \cellcolor{yellow!15}\\\cdashline{1-2}
\hspace{1mm}\emph{\% Internet}          & {\cellcolor{yellow!15}78\%} & & \cellcolor{yellow!15}\\\cdashline{1-2}
\hspace{1mm}\emph{\% Electricity}       & {\cellcolor{red!15}27\%} & & \cellcolor{yellow!15}\\\cdashline{1-2}
\hspace{1mm}\emph{\% Disability access} & {\cellcolor{yellow!15}75\%} & \multirow{-6}{4.2cm}{Inputs and infrastructure monitoring} & \multirow{-6}{*}{\cellcolor{yellow!15}3.1}\\\hline
\end{tabular}}
\\
{\scriptsize
    \textcolor{darkgray}{\textit{Notes:} \% refers to \% schools with the given sub-component}
  }

\end{table}

\hypertarget{school-management-by-principals}{%
\paragraph{\texorpdfstring{\textbf{School Management by
principals}}{School Management by principals}}\label{school-management-by-principals}}

In school management, the lowest score is for principal's Instructional
Leadership(3.4/5), whereas the highest score is obtained for Principal
Management Skills(4.2/5). School Management- Support policy lever scores
the lowest(3.7/5).

\begin{table}[H]
\resizebox{\columnwidth}{!}{\begin{tabular}{m{4cm}cm{4cm}c}
\multicolumn{2}{c}{\textbf{Practice Indicators}} & \multicolumn{2}{c}{\textbf{Policy levers(Management)}}\\\hline
Operational management                       & \cellcolor{green!15}4                   &                                       & \cellcolor{green!15}\\\cdashline{1-2}
Instructional Leadership                     & \cellcolor{yellow!15}3.4                   & \multirow{-2}{*}{Clarity of functions} & \multirow{-2}{*}{\cellcolor{green!15}4.9} \\\hline
                                             & \cellcolor{green!15}                    & Attraction                            & \cellcolor{green!15}4.3\\\cdashline{3-4}
\multirow{-2}{*}{Principal School knowledge}  & \multirow{-2}{*}{\cellcolor{green!15}4} & Selection \& Deployment               & \cellcolor{yellow!15}3.9\\\cdashline{3-4}
                                              & \cellcolor{green!15}                    & Support                               & \cellcolor{yellow!15}3.7\\\cdashline{3-4}
\multirow{-2}{*}{Principal Management skills}  & \multirow{-2}{*}{\cellcolor{green!15}4.2} & Evaluation                            & \cellcolor{green!15}4.5\\\hline
\end{tabular}}
\\
{\scriptsize
    \textcolor{darkgray}{\textit{Notes:} All scores are on a (0-5) scale.}
  }

\end{table}

\hypertarget{politics-bureaucratic-capacity-indicators}{%
\subsubsection{\texorpdfstring{\textbf{POLITICS \& BUREAUCRATIC CAPACITY
INDICATORS}}{POLITICS \& BUREAUCRATIC CAPACITY INDICATORS}}\label{politics-bureaucratic-capacity-indicators}}

Politics and bureaucratic capacity indicators measure the capacity and
orientation of the bureaucracy, as well as political factors affecting
education outcomes. The highest score in politics and bureaucratic
capacity is noted for Mandates \& Accountability (4.5/5), and the lowest
score is noted for Financing (2.4/5).

\begin{table}[H]
\resizebox{\columnwidth}{!}{\begin{tabular}{p{0.4\textwidth}c}
\textbf{Indicator} & \textbf{Value}\\ \hline
Quality of Bureaucracy & \cellcolor{green!15}4.4\\\cdashline{1-2}
Impartial Decision-Making & \cellcolor{yellow!15}3.7\\\cdashline{1-2}
Mandates \& Accountability & \cellcolor{green!15}4.5\\\cdashline{1-2}
National Learning Goals & \cellcolor{green!15}4.2\\\cdashline{1-2}
Financing & \cellcolor{red!15}2.4\\\hline
\end{tabular}}
\\
{\scriptsize
    \textcolor{darkgray}{\textit{Notes:} All scores are on a (0-5) scale.}
  }

\end{table}
\raggedbottom

{\scriptsize
    \textcolor{darkgray}{\textit{Disclaimer:} GEPD numbers presented in this brief are based on multiple sources including GEPD instruments, UIS, GLAD and Learning Poverty indicators. For that reason, the numbers discussed here may be different from official statistics reported by governments and national offices of statistics. Such differences are due to the different purposes of the statistics, which can be for global comparison or to meet national definitions.}
  }

\end{document}
